%!TEX encoding = UTF-8 Unicode
\documentclass[a4paper]{compendium}
\usepackage[swedish]{babel}
\addto\captionsswedish{%
  \renewcommand{\appendixname}{Appendix}%
}
%TODO: Glossary
%http://tex.stackexchange.com/questions/5821/creating-a-standalone-glossary/5837#5837

\setlength{\columnsep}{16mm}

\title{
{\bf\Huge\sffamily  Programmering, grundkurs}
\\ \vspace{2em}
{\sffamily  Kompendium}
}

%\author{Redaktör: Björn Regnell}
\date{EDAA45, Lp1-2, HT 2016 \\
Datavetenskap, LTH \\
Lunds Universitet  \\~\\
\url{http://cs.lth.se/pgk}}

\usepackage{pgffor}  %% http://stackoverflow.com/questions/2561791/iteration-in-latex
                     %  allows:  \foreach \n in {1,...,4}{ do something with \n }

\usepackage{framed}  %  allows:   \begin{framed}\end{framed}
%\newenvironment{Slide}[2][]
%  {\begin{framed}\setlist{noitemsep}\section*{#2}}
%  {\end{framed}}

\newcommand{\SlideHeading}[1]{\section*{#1}}

\usepackage[most]{tcolorbox}
\newenvironment{Slide}[2][]
  {\vspace{0.5em}\begin{tcolorbox}[%breakable,
                                   enhanced]\setlist{noitemsep}\SlideHeading{#2}}
  {\end{tcolorbox}\vspace{0.5em}}

\newcommand{\Subsection}[1]{} %ignore slide sections
\newcommand{\SlideOnly}[1]{} %ignore slide font size

\newif\ifkompendium  % to allow conditional text in slides only showing up in compendium
\kompendiumtrue      % in slides: \kompendiumfalse


\input{generated/names-generated.tex}

\begin{document}
\maketitle
\input{prechapters/licence-contributors.tex}
\input{prechapters/progress-forms.tex}
\input{prechapters/preface.tex}
\mainmatter
\tableofcontents

\part{Om kursen}
\input{prechapters/course-architecture.tex}
\input{prechapters/course-instructions.tex}
\input{prechapters/how-to-contribute.tex}

\renewcommand{\SlideHeading}[1]{\section{#1}}

\part{Moduler}
%!TEX encoding = UTF-8 Unicode
%!TEX root = ../compendium.tex

\input{generated/w01-chaphead-generated.tex}
\clearpage

\input{../slides/body/lect-week01-intro.tex}








%!TEX encoding = UTF-8 Unicode
%!TEX root = ../compendium.tex

\Exercise{\ExeWeekONE}\label{exe:W01}

\begin{Goals}
\item Förstå vad som händer när satser exekveras och uttryck evalueras.
\item Förstå sekvens, alternativ och repetition.
\item Känna till literalerna för enkla värden, deras typer och omfång.
\item Kunna deklarera och använda variabler och tilldelning, samt kunna rita bilder av minnessituationen då variablers värden förändras.
\item Förstå skillnaden mellan olika numeriska typer, kunna omvandla mellan dessa och vara medveten om noggranhetsproblem som kan uppstå.
\item Förstå booelska uttryck och värdena \code{true} och \code{false}, samt kunna förenkla booelska uttryck.
\item Förstå skillnaden mellan heltalsdivision och flyttalsdivision, samt använding av rest vid heltalsdivision.
\item Förstå precedensregler och användning av parenteser i uttryck.
\item Kunna använda \code{if}-satser och \code{if}-uttryck.
\item Kunna anvädna \code{for}-satser och \code{while}-satser.
\item Kunna använda \code{math.random} för att generera slumptal i olika interval.
\end{Goals}

\begin{Preparations}
\item Studera teorin i kapitel~\ref{chapter:W01}.
\item Du behöver en dator med Scala installerad; se appendix~\ref{appendix:compile}.
\end{Preparations}

\BasicTasks

\Task Starta Scala REPL \Eng{Read-Evaluate-Print-Loop} och skriv satsen \code{println("hejsan REPL")} och tryck på \textit{Enter}. 

\begin{REPLnonum}
> scala
Welcome to Scala version 2.11.7 (Java HotSpot(TM) 64-Bit Server VM, Java 1.8).
Type in expressions to have them evaluated.
Type :help for more information.

scala> println("hejsan REPL")
\end{REPLnonum}

\Subtask Vad händer? 

\Subtask Skriv samma sats igen men ''glöm bort'' att skriva högerparentesen innan du trycker på \textit{Enter}. Vad händer?

\Subtask Evaulera uttrycket \code{"gurka" + "tomat"} i REPL. Vad har uttrycket för värde och typ? Vilken siffra står efter ordet \code{res} i variabeln som lagrar resultatet?

\begin{REPLnonum}
scala> "gurka" + "tomat"   
\end{REPLnonum}

\Subtask Evaluera uttrycket \code{res0 * 42} men byt ut \code{0}:an mot siffran efter \code{res} i utskriften från förra evalueringen. Vad har uttrycket för värde och typ?
\begin{REPLnonum}
scala> res2 * 42
\end{REPLnonum}


\Task\Pen Vad är en \textit{literal}? \\ \href{https://en.wikipedia.org/wiki/Literal\_\%28computer_programming\%29}{en.wikipedia.org/wiki/Literal\_(computer programming)}

\Task Vilken typ har följande literaler?

\Subtask \code{42} 

\Subtask \code{42L}

\Subtask \code{'*'}

\Subtask \code{"*"}

\Subtask \code{42.0}

\Subtask \code{42D}

\Subtask \code{42d}

\Subtask \code{42F}

\Subtask \code{42f}

\Subtask \code{true}

\Subtask \code{false}


\Task\Pen Vad gör dessa satser? Till vad används klammer och semikolon?
\begin{REPLnonum}
scala> def p = { print("hej"); print("san"); println(42); println("gurka") }
scala> p;p;p;p
\end{REPLnonum}

\Task\Pen Satser versus uttryck. 

\Subtask Vad är det för skillnad på en sats och ett uttryck?

\Subtask Ge exempel på satser som inte är uttryck?

\Subtask Förklara vad som händer för varje evaluerad rad:
\begin{REPL}
scala> def värdeSaknas = ()
scala> värdeSaknas
scala> värdeSaknas.toString
scala> println(värdeSaknas)
scala> println(println("hej"))
\end{REPL}

\Subtask Vilken typ har literalen \code{()}?

\Subtask Vilken returtyp har \code{println}?

\Task Vilken typ och vilket värde har följande uttryck? 

\Subtask \code{1 + 41}

\Subtask \code{1.0 + 41}

\Subtask \code{42.toDouble}

\Subtask \code{(41 + 1).toDouble}

\Subtask \code{1.042e42}

\Subtask \code{42E6.toLong}

\Subtask \code{"gurk" + 'a'}

\Subtask \code{'A'}

\Subtask \code{'A'.toInt}

\Subtask \linebreak[0] \code{'0'.toInt}

\Subtask \code{'1'.toInt}

\Subtask \code{'9'.toInt}

\Subtask \code{('A' + '0').toChar}

\Subtask \code{"*!%#".charAt(0)}

\Task \textit{De fyra räknesätten}. Vilket värde och vilken typ har följande uttryck?

\Subtask \code{42 * 2}

\Subtask \code{42.0 / 2}

\Subtask \code{42 - 0.2}

\Subtask \code{42L + 2d}

\Task \textit{Precedensregler}. Evalueringsordningen kan styras med parenteser. Vilket värde och vilken typ har följande uttryck?

\Subtask \code{42 + 2 * 2}

\Subtask \code{(42 + 2) * 2}

\Subtask \code{(-(2 - 42)) / (1 + 1 + 1).toDouble}

\Subtask \code{((-(2 - 42)) / (1 + 1 + 1).toDouble).toInt}


\Task \textit{Heltalsdivision}. Vilket värde och vilken typ har följande uttryck?

\Subtask \code{42 / 2}

\Subtask \code{42 / 4}

\Subtask \code{42.0 / 4}

\Subtask \code{1 / 4}

\Subtask \code{1 % 4}

\Subtask \code{2 % 42}

\Subtask \code{42 % 2}


\Task \textit{Hetalsomfång}. För var och en av heltalstyperna i deluppgifterna nedan: undersök i REPL med operationen \code{MaxValue} resp. \code{MinValue}, vad som är största och minsta värde, till exempel \code{Int.MaxValue} etc. 

\Subtask \code{Byte}

\Subtask \code{Short}

\Subtask \code{Int}

\Subtask \code{Long}

\Task Klassen \code{java.lang.Math} och paketobjektet \code{scala.math}.

\begin{REPL}
scala> java.lang.Math.    //tryck TAB
scala> scala.math.        //tryck TAB
\end{REPL}

\Subtask Undersök genom att trycka på Tab-tangenten, vilka funktioner som finns i \code{Math} och \code{math}. Vad heter konstanten $\pi$ i java.lang.Math respektive scala.math?


\Subtask Undersök dokumentationen för klassen \code{java.lang.Math} här: \\ \url{https://docs.oracle.com/javase/8/docs/api/java/lang/Math.html} \\
Vad gör \code{java.lang.Math.hypot}?

\Subtask Undersök dokumentationen för pakobjektet \code{scala.math} här: \\
\url{http://www.scala-lang.org/api/current/#scala.math.package} \\
Ge exempel på någon funktion i \code{java.lang.Math} som inte finns i \code{scala.math}.

%\TaskSection{Noggranhet och undantag i aritmetiska uttryck}

\Task Vad händer här? Notera undantag \Eng{exceptions} och nogranhetsproblem. 

\Subtask \code{Int.MaxValue} + 1

\Subtask \code{1 / 0}

\Subtask \code{1E8 + 1E-8}

\Subtask \code{1E9 + 1E-9}

\Subtask \code{math.pow(math.hypot(3,6), 2)}

\Subtask \code{1.0 / 0}

\Subtask \code{(1.0 / 0).toInt}

\Subtask \code{math.sqrt(-1)}

\Subtask \code{math.sqrt(Double.NaN)}

\Subtask \code{throw new Exception("PANG!!!")}


\Task \textit{Booelska uttryck}. Vilket värde och vilken typ har följande uttryck?

\Subtask \code{true && true}

\Subtask \code{false && true}

\Subtask \code{true && false}

\Subtask \code{false && false}

\Subtask \code{true || true}

\Subtask \code{false || true}

\Subtask \code{true || false}

\Subtask \code{false || false}

\Subtask \code{42 == 42}

\Subtask \code{42 != 42}

\Subtask \code{42.0001 == 42}

\Subtask \code{42.0000000000000001 == 42}

\Subtask \code{42.0001 > 42}

\Subtask \code{42.0000000000000001 > 42}

\Subtask \code{42.0001 >= 42}

\Subtask \code{42.0000000000000001 <= 42}

\Subtask \code{true == true}

\Subtask \code{true != true}

\Subtask \code{true > false}

\Subtask \code{true < false}

\Subtask \code{'A' == 65}

\Subtask \code{'S' != 66}


\Task\Pen \textit {Variabler och tilldelning}. Rita en ny bild av datorns minne efter varje evaluerad rad nedan. Bilderna ska visa variablers namn, typ och värde. 
\begin{REPL}[numbers=left, numberstyle=\color{black}\ttfamily\scriptsize\selectfont]
scala> var a = 42
scala> var b = a + 1
scala> var c = (a + b) * 2.0
scala> b = 0
scala> a = 0
scala> c = c + 1
\end{REPL}
Efter första raden ser minnessituationen ut så här:

\vspace{0.5em}
\begin{tikzpicture}[font=\ttfamily]
\matrix [matrix of nodes, row sep=0, column 2/.style={nodes={rectangle,draw,minimum width=4em}}] (mat)
{
a: Int   &  \makebox(16,12){42}\\
};
\end{tikzpicture}

\Task \textit{Deklarationer: \code{var}, \code{val}, \code{def}}. Evaluera varje rad nedan i tur och ordning i Scala REPL. 
\begin{REPL}[numbers=left, numberstyle=\color{black}\ttfamily\scriptsize\selectfont]
scala> var x = 42
scala> x + 1
scala> x
scala> x = x + 1
scala> x
scala> x == x + 1
scala> val y = 42
scala> y = y + 1
scala> var z = {println("gurka"); 42}
scala> def w = {println("gurka"); 42}
scala> z
scala> z
scala> z = z + 1
scala> w
scala> w
scala> w = w + 1
\end{REPL}

\Subtask För varje rad ovan: förklara för vad som händer. 

\Subtask Vilka rader ger kompileringsfel och i så fall vilket och varför?

\Subtask\Pen Vad är det för skillnad på \code{var}, \code{val} och \code{def}?

\Task \code{if}\textit{-sats}.För varje rad nedan; förklara vad som händer.
\begin{REPL}
scala> if (true) println("sant") else println("falskt")
scala> if (false) println("sant") else println("falskt")
scala> if (!true) println("sant") else println("falskt")
scala> if (!false) println("sant") else println("falskt")
scala> def kasta = if (math.random > 0.5) println("krona") else println("klave")
scala> kasta; kasta; kasta
\end{REPL}


\Task \code{if}\textit{-uttryck}. Deklarera följande variabler med nedan initialvärden:

\begin{REPLnonum}
scala> var grönsak = "gurka"
scala> var frukt = "banan"
\end{REPLnonum}

Vad har följande uttryck för värden och typ?

\Subtask \code{if (grönsak == "tomat") "gott" else "inte gott" }

\Subtask \code{if (frukt == "banan") "gott" else "inte gott" }

\Subtask \code{if (frukt.size == grönsak.size) "lika stora" else "olika stora" }

\Subtask \code{if (true) grönsak else frukt }

\Subtask \code{if (false) grönsak else frukt }


\Task \code{for}\textit{-sats}.

\Subtask Vad ger nedan \code{for}-satser för utskrift?

\begin{REPL}
scala> for (i <- 1 to 10) print(i + ", ")
scala> for (i <- 1 until 10) print(i + ", ")
scala> for (i <- 1 to 5) print((i * 2) + ", ")
scala> for (i <- 1 to 92 by 10) print(i + ", ")
scala> for (i <- 10 to 1 by -1) print(i + ", ")
\end{REPL}

\Subtask Skriv en \code{for}-sats som ger följande utskrift:
\begin{REPLnonum}
A1, A4, A7, A10, A13, A16, A19, A22, A25, A28, A31, A34, A37, A40, A43, 
\end{REPLnonum}

\Task Repetition med \code{foreach}.

\Subtask Vad ger nedan satser för utskrifter?

\begin{REPL}
scala> (9 to 19).foreach{i => print(i + ", ")}
scala> (1 until 20).foreach{i => print(i + ", ")}
scala> (0 to 33 by 3).foreach{i => print(i + ", ")}
\end{REPL}

\Subtask Använd \code{foreach} och skriv ut följande:
\begin{REPLnonum}
B33, B30, B27, B24, B21, B18, B15, B12, B9, B6, B3, B0, 
\end{REPLnonum}

\Task \code{while}\textit{-sats}. 

\Subtask Vad ger nedan satser för utskrifter?
\begin{REPL}
scala> var i = 0
scala> while (i < 10) { println(i); i = i + 1 }
scala> var j = 0; while (j <= 10) { println(j); j = j + 2 }; println(j)
\end{REPL}

\Subtask Skriv en \code{while}-sats som ger följande utskrift. Använd en variabel \code{k} som initialiseras till 1.
\begin{REPLnonum}
A1, A4, A7, A10, A13, A16, A19, A22, A25, A28, A31, A34, A37, A40, A43, 
\end{REPLnonum}

\Subtask\Pen Vilken av \code{for}, \code{while} och \code{foreach} är kortast att skriva om man vill repetera mer än en sats 100 gånger? Vilken tycker du är lättast att läsa? 

\Task \textit{Slumptal}. Undersök vad dokumentationen säger om funktionen \code{scala.math.random}:\\
\url{http://www.scala-lang.org/api/current/#scala.math.package} 

\Subtask\Pen Vilken typ har värdet som returneras av funktionen \code{random}? 
 
\Subtask\Pen Vilket är det minsta respektive största värde som kan returneras? 

\Subtask\Pen Är \code{random} en \textit{äkta} funktion \Eng{pure function} i matematisk mening?

\Subtask Anropa funktionen \code{math.random} upprepade gånger och notera vad som händer. Använd pil-upp-tangenten.
\begin{REPLnonum}
scala> math.random
\end{REPLnonum}


\Subtask Vad händer? Använd \textit{pil-upp} och kör nedan \code{for}-sats flera gånger. Förklara vad som sker.

\begin{REPLnonum}
scala> for (i <- 1 to 10) println(math.random)
\end{REPLnonum}

\Subtask Skriv en for-sats som skriver ut 100 slumpmässiga heltal från 0 till och med 9 på var sin rad. 

\begin{REPLnonum}
scala> for (i <- 1 to 100) println(???)
\end{REPLnonum}

\Subtask Skriv en for-sats som skriver ut 100 slumpmässiga heltal från 1 till och med 6 på samm rad. 

\begin{REPLnonum}
scala> for (i <- 1 to 100) print(???)
\end{REPLnonum}


\Subtask Använd \textit{pil-upp} och kör nedan \code{while}-sats flera gånger. Förklara vad som sker.

\begin{REPLnonum}
scala> while (math.random > 0.2) println("gurka")
\end{REPLnonum}

\Subtask Ändra i \code{while}-satsen ovan så att sannolikheten ökar att riktigt många  strängar ska skrivs ut.

\Subtask Förklara vad som händer nedan.
\begin{REPL}
scala> var slumptal = math.random
scala> while (slumptal > 0.2) { println(slumptal); slumptal = math.random }
\end{REPL}

\Task\Pen \textit{Logik och De Morgans Lagar}. Förenkla följande uttryck. Antag att \code{poäng} och \code{highscore} är heltalsvariabler medan \code{klar} är av typen \code{Boolean}. 

\Subtask \code{poäng > 100 && poäng > 1000}

\Subtask \code{poäng > 100 || poäng > 1000}

\Subtask \code{!(poäng > highscore)}

\Subtask \code{!(poäng > 0 && poäng < highscore) }

\Subtask \code{!(poäng < 0 || poäng > highscore) }

\Subtask \code{klar == true}

\Subtask \code{klar == false}


\ExtraTasks

\Task \textit{Slumptal}.

\Subtask Ersätt \code{???} nedan med literaler så att \code{tärning} returnerar ett slumpmässigt heltal mellan 1 och 6.
\begin{REPLnonum}
scala> def tärning = (math.random * ??? + ???).toInt 
\end{REPLnonum}

\Subtask Ersätt \code{???} med literaler så att rnd blir ett decimaltal med max en decimal mellan 0.0 och 1.0.
\begin{REPLnonum}
scala> def rnd = math.round(math.random * ???) / ??? 
\end{REPLnonum}

\Subtask Vad blir det för skillnad om \code{math.round} ersätts med \code{math.floor} ovan? (Se dokumentationen av \code{java.lang.Math.round} och \code{java.lang.Math.floor}.)

\AdvancedTasks

\Task Läs om moduloräkning här \href{https://en.wikipedia.org/wiki/Modulo\_operation}{en.wikipedia.org/wiki/Modulo\_operation} och undersök hur tecknet blir med olika tecken på divisor och dividend.

\Task Integer.toBinaryString, Integer.toHexString

\Task Typannoteringar.

\Task 0x2a

\Task \code{i += 1; i *= 1; i /= 2}

\Task BigInt, BigDecimal

\Task Vad händer här? 
\begin{REPLnonum}
scala> Math.multiplyExact(2, 42)
scala> Math.multiplyExact(Int.MaxValue, Int.MaxValue)
\end{REPLnonum}

\Task Sök reda på dokumentationen i javadoc för klassen \code{java.lang.Math} i JDK 8. Tryck Ctrl+F i webbläsaren och sök efter förekomster av texten \textit{''overflow''}. Vad är ''overflow''? Vilka metoder finns i \code{java.lang.Math} som hjälper dig att upptäcka om det blir overflow?

\Task Använda Scala REPL för att undersöka konstanterna nedan. Vilket av dessa värden är negativt? Vad kan man ha för praktisk nytta av dessa värden i ett program som gör flyttalsberäkningar?

\Subtask \code{java.lang.Double.MIN_VALUE}

\Subtask \code{scala.Double.MinValue} 

\Subtask \code{scala.Double.MinPositiveValue}

\Task För typerna \code{Byte}, \code{Short}, \code{Char}, \code{Int}, \code{Long}, \code{Float}, \code{Double}: Undersök hur många bitar som behövs för att representera varje typs omfång? \\*
\textit{Tips:} Några användbara uttryck: \\*
 \code{Integer.toBinaryString(Int.MaxValue + 1).size} \\*
 \code{Integer.toBinaryString((math.pow(2,16) - 1).toInt).size} \\*
 \code{1 + math.log(Long.MaxValue)/math.log(2)}
Se även språkspecifikationen för Scala, kapitlet om heltalsliteraler: \\
\url{http://www.scala-lang.org/files/archive/spec/2.11/01-lexical-syntax.html#integer-literals}

\Subtask Undersök källkoden för pakobjektet \code{scala.math} här: \\
\url{https://github.com/scala/scala/blob/v2.11.7/src/library/scala/math/package.scala} \\
Hur många olika överlagrade varianter av funktionen \code{abs} finns det och för vilka parametertyper är den definierad?



%!TEX encoding = UTF-8 Unicode
%!TEX root = ../compendium.tex

\Lab{\LabWeekONE}

\begin{Goals}
\item Kunna kombinera principerna sekvens, alternativ, repetition, och abstraktion i skapandet av egna program om minst 20 rader kod.
\item Kunna förklara vad ett program gör i termer av sekvens, alternativ, repetition, och abstraktion.
\item Kunna tillämpa principerna sekvens, alternativ, repetition, och abstraktion i enkla algoritmer.
\item Kunna formatera egna program så att de blir lätta att läsa och förstå.
\item Kunna förklara vad en variabel är och kunna skriva deklarationer och göra tilldelningar.
\item Kunna genomföra upprepade varv i cykeln \emph{editera-exekvera-felsöka/förbättra} för att succesivt bygga upp allt mer utvecklade program. 
\end{Goals}

\begin{Preparations}
\item Gör övning {\tt \ExeWeekONE} i kapitel \ref{exe:W01}.
\item Läs igenom ''Kojo - An Introduction'' (25 sidor) som du kan ladda ner i pdf  här: \href{http://www.kogics.net/kojo-ebooks}{http://www.kogics.net/kojo-ebooks}
\item Du behöver en dator med Kojo installerad, se appendix \ref{appendix:kojo}.
\end{Preparations}

\subsection{Obligatoriska uppgifter}


\Task \textit{Sekvens}. 

\Subtask Starta Kojo. Om du inte redan har svenska menyer: välj svenska i språkmenyn och starta om Kojo.  Skriv in nedan program och tryck på den \emph{gröna} play-knappen. Du hittar en lista med några fler funktioner på svenska och engelska i appendix \ref{appendix:kojo}.

\begin{Code}
sudda

fram; höger
fram; vänster
\end{Code}


\Subtask Prova att ändra på ordningen mellan satserna och använd den \emph{gula} play-knappen  (programspårning) för att studera vad som händer. Klicka på satser i ditt program och på rutor i programspårningen och se vad som händer.

\Subtask Prova satser i sekvens på flera rader, respektive på samma rad med semikolon emellan. Hur vill du gruppera dina satser så att de lätta för en människa att läsa?

\Subtask\Pen Vad händer om du \emph{inte} börjar programmet med \code{sudda} och kör det upprepade gånger? Varför är det bra börja programmet med \code{sudda}? 

\Subtask Rita en kvadrat som i bilden nedan.

\includegraphics{../img/kojo/kvadrat}

\Subtask Rita en trappa som i bilden nedan.

\includegraphics[width=0.25\textwidth]{../img/kojo/stairs}

\Subtask \emph{Rita och mät}. 
\begin{itemize}[noitemsep]
\item Börja ditt program med dessa satser:\\ \code{sudda; axesOn; gridOn; sakta(0); osynlig} 
\item Rita sedan en kvadrat som har 444 längdenheter i omkrets. 
\item Ta fram linjalen med höger-klick i ritfönstret och mät så exakt du kan hur lång diagonalen i kvadraten är. Skriv ner resultatet. \\ \emph{Tips:} Du kan zooma med mushjulet om du håller nere Ctrl-knappen. Du kan flytta linjalen om du klick-drar på linjalens skalstreck. Du kan vrida linjalen om du klickar på skalstrecken och håller nere Shift-tangenten. 
\item Kontrollera med hjälp av \code{math.hypot} och \code{println} vad det exakta svaret är. Skriv ner svaret med 3 decimalers noggrannhet. 
\end{itemize}

\Subtask Rita en triangel med sidan 300 längdenheter genom att ge lämpliga argument till \code{fram} och \code{höger}. Vinklar anges i grader.

\Subtask\Checkpoint Visa dina resultat för en handledare och diskutera hur uppgifterna ovan illustrerar principen om sekvens.

\Task \textit{Repetition}. 

\Subtask Rita en kvadrat igen, men nu med hjälp av proceduren \code+upprepa(n){ ??? }+ där du ersätter n med antalet repetitioner och ??? med de satser som ska repeteras. 

\Subtask Kör ditt program med den \emph{gula} play-knappen. Studera hur repetitionen påverkar  exekveringssekvensen. Vid vilka punkter i programmet sker ett ''hopp'' i sekvensen i stället för att efterföljande sats att exekveras? Använd lämpligt argument till \code{sakta} för att du ska hinna studera exekveringen.

\Subtask Anropa proceduren \code{sakta(???)} med lämplig parameter och gör så att sköldpaddan går totalt 20 varv i kvadraten på ungefär 2 sekunder. \emph{Tips:} Du kan köra ditt program med \emph{Ctrl+Enter} i stället för att trycka på den gröna play-knappen. Anropa \code{sakta} i början av ditt program men \emph{efter} sudda. (Vad händer om du anropar \code{sakta} före \code{sudda}?)


\Subtask Om du anropar \code{sakta(0)}, hur många kvadratvarv hinner sköldpaddan rita på en sekund? Använd nedan program för att ta reda på ungefärligt antal varv per sekund.
\begin{Code}
sudda; sakta(0)
val t1 = System.currentTimeMillis
upprepa(800*4){fram;höger}
val t2 = System.currentTimeMillis
println("Det tog " + (t2 - t1) + " millisekunder")
\end{Code} 



\Subtask Rita en kvadrat igen, men nu med hjälp av en \code{while}-sats och en loop-variabel.

\begin{Code}
var i = 0
while (???) {fram; höger; i = ???}
\end{Code} 

\Subtask Rita en kvadrat igen, men nu med hjälp av en \code{for}-sats.

\begin{Code}
for (i <- 1 to ???) {???}
\end{Code} 

\Subtask Rita en kvadrat igen, men nu med hjälp av \code{foreach}.

\begin{Code}
(1 to ???).foreach{i => ???}
\end{Code} 


\Subtask\Checkpoint Vad är fördelar och nackdelar med de olika sätten att loopa: \code{upprepa}, \code{while}, \code{for}, respektive \code{foreach}? \Pen Diskutera dina svar med en handledare.

\Task \textit{Abstraktion}. 

\Subtask Använd en repetition för abstrahera nedan sekvens, så att programmet blir kortare:
\begin{Code}
sudda

fram; höger; hoppa; fram; vänster; hoppa; fram; höger; 
hoppa; fram; vänster; hoppa; fram; höger; hoppa; fram; 
vänster; hoppa; fram; höger; hoppa; fram; vänster; hoppa;
fram; höger; hoppa; fram; vänster; hoppa 
\end{Code}

\Subtask\Pen Sök på nätet efter ''DRY principle programming'' och beskriv med egna ord vad DRY betyder och varför det är en viktig princip.

\Subtask Använd proceduren \code{kvadrat} nedan och proceduren \code{hoppa(???)} för att rita en stapel med 10 kvadrater enligt bilden.

\begin{Code}
def kvadrat = for (i <- 1 to 4) {fram; höger}
\end{Code}

\includegraphics[scale=0.5]{../img/kojo/square-column}

\Subtask Kör ditt program med den \emph{gula} play-knappen. Studera hur anrop av proceduren \code{kvadrat} påverkar exekveringssekvensen av dina satser. Vid vilka punkter i programmet sker ett ''hopp'' i sekvensen i stället för att efterföljande sats att exekveras? Använd lämpligt argument till \code{sakta} för att du ska hinna studera exekveringen.

\Subtask Rita samma bild med 10 staplade kvadrater som ovan, men nu \emph{utan} att använda abstraktionen \code{kvadrat} -- använd i stället en nästlad repetition. Vilket av de två sätten (med och utan abstraktionen \code{kvadrat}) är lättast att läsa?\\ \emph{Tips:} Varje gång du trycker på någon av play-knapparna, sparas ditt program. Du kan se dina sparade program om du klickar på \emph{Historik}-fliken. Du kan också stega bakåt och framåt i historiken med de blå pilarna bredvid play-knapparna. 

\Subtask Skapa en abstraktion \code{def stapel = ???} med din kod för att rita en stapel.

\Subtask Du ska nu generalisera din procedur så att den inte bara kan rita exakt 10 kvadrater i en stapel. Ge proceduren \code{stapel} en parameter \code{n} som styr hur många kvadrater som ritas.
\begin{Code}
def kvadrat = ???
def stapel(n: Int) = ???

sudda; sakta(100)
stapel(42)
\end{Code}

\Subtask Ge abstraktionen \code{kvadrat} en parameter \code{sida: Double} som anger hur stor kvadraten blir. Rita flera kvadrater i likhet med bilden nedan.

\includegraphics{../img/kojo/square-param}

\Subtask Rita nedan bild med hjälp av abstraktionen \code{stapel}. Det är totalt 100 kvadrater och varje kvadrat har sidan 25. \emph{Tips:} Med ett negativt argument till procedure \code{hoppa} kan du få sköldpaddan att hoppa baklänges utan att rita, t.ex. \code{hoppa(-10*25)}

\includegraphics[width=0.3\textwidth]{../img/kojo/square-grid}

\Subtask Skapa en abstraktion \code{rutnät} med lämpliga parametrar som gör att man kan rita rutnät med olika stora kvadrater och olika många kvadrater i både x- och y-led.

\Subtask\Checkpoint Se över ditt program i föregående uppgift och säkerställ att det är lättläst och följer en struktur som börjar med alla definitioner i logisk ordning och därefter fortsätter med huvudprogrammet. Diskutera ditt program med en handledare. Vad har du gjort för att programmet ska vara lättläst?


\Task \emph{Variabel.}

\Subtask Skriv in nedan program \emph{exakt} som det står med blanktecken, indragningar och radbrytningar. Kör programmet och förklara vad som händer.

\begin{Code}
def gurka(x: Double,
          y: Double, namn: String,
          typ: String,
          värde:String) = {
  val bredd = 15
  val h = 30
  hoppaTill(x,y)
  norr
  skriv(namn+": "+typ)
  hoppaTill(x+bredd*(namn.size+typ.size),y)
  skriv(värde); söder; fram(h); vänster
  fram(bredd * värde.size); vänster
  fram(h); vänster
  fram(bredd * värde.size); vänster
}

sudda; färg(svart)
val s = 130
val h = 40
var x = 42; gurka(10, s-h*0, "x","Int", x.toString)
var y = x;  gurka(10, s-h*1, "y","Int", y.toString)
x = x + 1;  gurka(10, s-h*2, "x","Int", x.toString)
            gurka(10, s-h*3, "y","Int", y.toString)
osynlig
\end{Code}

\Subtask\Pen Skriv ner namnet på alla variabler som förekommer i programmet ovan. 

\Subtask\Pen Vilka av dessa variabler är lokala? 

\Subtask\Pen Vilka av dessa variabler kan förändras?

\Subtask\Pen Föreslå tre förändringar av programmet ovan (till exempel namnbyten) som gör att det blir lättare att läsa och förstå.

\Subtask Gör sök-ersätt av \code{gurka} till ett bättre namn. \emph{Tips:} undersök kontextmenyn i editorn i Kojo genom att högerklicka i editorfönstret. Notera kortkommandot för Sök/Ersätt.

\Subtask\Checkpoint Gör automatisk formattering av koden med hjälp av lämpligt editor-kortkommando. Notera skillnaderna. Vilket autoformatteringar gör programmet lättare att läsa? Vilka manuella formatteringar  tycker du bör göras för att öka läsbarheten? Diskutera läsbarheten med en handledare. 

\Task \emph{Alternativ.}

\Subtask Kör programmet nedan. Förklara vad som händer. Använd den gula play-knappen för att studera exekveringen.

\begin{Code}
sudda; sakta(5000)

def move(key: Int): Unit = {
  println("key: " + key)
  if (key == 87) fram(10) 
  else if (key == 83) fram(-10)  
}

move(87); move('W'); move('W')
move(83); move('S'); move('S'); move('S')
\end{Code}

\Subtask Kör programmet nedan. Notera \code{activateCanvas} för att du ska slippa klicka i ritfönstret innan du kan styra paddan. Lägg till kod i \code{move} som gör att tangenten A ger en vridning moturs med 5 grader medan tangenten D ger en vridning medurs 5 grader.

\begin{Code}
sudda; sakta(0); activateCanvas

def move(key: Int): Unit = {
  println("key: " + key)
  if (key == 'W') fram(10) 
  else if (key == 'S') fram(-10)  
}

onKeyPress(move)
\end{Code}

\Subtask Lägg till nedan kod i början av programmet och gör så att när man trycker på tangenten G så  sätter man omväxlande på och av rutnätet. 

\begin{Code}
var isGridOn = false

def toggleGrid = 
  if (isGridOn) {
    gridOff
    isGridOn = false
  } else {
    gridOn
    isGridOn = true  
  }
\end{Code}

\Subtask\Checkpoint Gör så att när man trycker på tangenten X så sätter man omväxlande på och av koordinataxlarna. Använd en variabel \code{isAxesOn} och definiera en abstraktion \code{toggleAxes} som anropar \code{axesOn} och \code{axesOff} på liknande sätt som i föregående uppgift. Visa din lösning för en handledare.

\subsection{Frivilliga extrauppgifter}

\Task \emph{Tidmätning.} Hur snabb är din dator?

\Subtask \label{task:timer} Skriv in koden nedan i Kojos editor och kör upprepade gånger med den gröna play-knappen. Hur långt tid tar det för din dator att räkna till 4.4 miljarder?\footnote{Det går att göra ungefär en heltalsaddition per klockcykel per kärna. Den första elektroniska datorn Eniac hade en klockfrekvens motsvarane 5kHz. Björn Regnells dator har en i7-4790K som turboklockar på 4.4 MHz. 

\href{http://www.extremetech.com/computing/185512-overclocking-intels-core-i7-4790k-can-devils-canyon-fix-haswells-low-clock-speeds/2}{www.extremetech.com/computing/185512-overclocking-intels-core-i7-4790k-can-devils-canyon-fix-haswells-low-clock-speeds/2}
}

\begin{Code}
object timer {
  def now: Long = System.currentTimeMillis
  var saved: Long = now
  def elapsedMillis: Long = now - saved
  def elapsedSeconds: Double = elapsedMillis / 1000.0
  def reset: Unit = { saved = now }
}

// HUVUDPROGRAM:
timer.reset
var i = 0L
while (i < 1e8.toLong) { i += 1 }
val t = timer.elapsedSeconds
println("Räknade till " + i + " på " + t + " sekunder.")
\end{Code}

\Subtask  Om du kör på en Linux-maskin: Kör nedan Linux-kommando upprepade gånger i ett terminalfönster. Med hur många MHz kör din dators klocka för tillfället? Hur förhåller sig klockfrekvensen till antalet rundor i while-loopen i föregående uppgift? (Det kan hända att din dator kan variera centralprocessorns klockfrekvens. Prova både medan du kör tidmätningen i Kojo och då din dator ''vilar''. Vad är det för poäng med att en processor kan variera sin klockfrekvens?)
\begin{REPL}
> lscpu | grep MHz
\end{REPL}


\Subtask Ändra i koden i uppgift \ref{task:timer}) så att \code{while}-loopen bara kör 5 gånger. Kör programmet med den \emph{gula} play-kappen. Scrolla i programspårningen och förklara vad som händer. Klicka på \code{CALL}-rutorna och se vilken rad som markeras i ditt program.

\Subtask Lägg till koden nedan i ditt program och försök ta reda på ungefär hur långt din dator hinner räkna till på en sekund för \code{Long}- respektive \code{Int}-variabler. Använd den gröna play-knappen.
\begin{Code}
def timeLong(n: Long): Double = {
  timer.reset
  var i = 0L
  while (i < n) { i += 1 }
  timer.elapsedSeconds
}

def timeInt(n: Int): Double = {
  timer.reset
  var i = 0
  while (i < n) { i += 1 }
  timer.elapsedSeconds
}

def show(msg: String, sec: Double): Unit = {
  print(msg + ": ")
  println(sec + " seconds")
}

def report(n: Long): Unit = {
  show("Long " + n, timeLong(n))
  if (n <= Int.MaxValue) show("Int  " + n, timeInt(n.toInt))
}

// HUVUDPROGRAM, mätningar:

report(Int.MaxValue)

for (i <- 1 to 10) {
  report(4.26e9.toLong)
}
\end{Code}

\Subtask\Checkpoint Hur mycket snabbare går det att räkna med \code{Int}-variabler jämfört med \code{Long}-variabler? Visa svaret för en handledare.

\Task Lek med färg i Kojo. Sök på internet efter dokumentationen för klassen java.awt.Color och studera vilka heltalsparametrar den sista konstruktorn i listan med konstruktorer tar för att skapa sRGB-färger. Om du högerklickar i editorn i Kojo och väljer ''Välj färg...'' får du fram färgväljaren.

\begin{REPL}
scala> val c = new java.awt.Color(124,10,78,100)
c: java.awt.Color = java.awt.Color[r=124,g=10,b=78]

scala> c.  // tryck på TAB
asInstanceOf    getColorComponents      getRGBComponents   
brighter        getColorSpace           getRed             
createContext   getComponents           getTransparency    
darker          getGreen                isInstanceOf       
getAlpha        getRGB                  toString           
getBlue         getRGBColorComponents                      

scala> c.getAlpha
res3: Int = 100
\end{REPL}

\includegraphics[width=0.75\textwidth]{../img/kojo/random-color-circles.png}


\Task Ladda ner dessa pdf-kompendier och gör några uppgifter som du tycker verkar intressanta:

\Subtask ''Uppdrag med Kojo'' som kan laddas ner här:\\ \href{http://fileadmin.cs.lth.se/cs/Personal/Bjorn_Regnell/uppdrag.pdf}{fileadmin.cs.lth.se/cs/Personal/Bjorn\_Regnell/uppdrag.pdf}

\Subtask ''Programming Fundamentals with Kojo'' som kan laddas ner här:\\
 \href{http://wiki.kogics.net/kojo-codeactive-books}{wiki.kogics.net/kojo-codeactive-books}
%!TEX encoding = UTF-8 Unicode
%!TEX root = ../compendium.tex

\input{generated/w02-chaphead-generated.tex}



%!TEX encoding = UTF-8 Unicode
%!TEX root = ../compendium.tex

\Exercise{\ExeWeekTWO}

\begin{Goals}
\item Kunna skapa samlingarna Range, Array och Vector med heltals- och strängvärden.
\item Kunna indexera i en indexerbar samling, t.ex. Array och Vector.
\item Kunna anropa operationerna size, mkString, sum, min, max på samlingar som innehåller heltal.
\item Känna till grundläggande skillnader och likheter mellan samlingarna Range, Array och Vector.
\item Förstå skillnaden mellan en for-sats och ett for-uttryck.
\item Kunna skapa samlingar med heltalsvärden som resultat av enkla for-uttryck.
\item Förstå skillnaden mellan en algoritm i pseudo-kod och dess implementation.
\item Kunna implementera algoritmerna SUM, MIN/MAX på en indexerbar samling med en \code{while}-sats.
\item Kunna köra igång enkel Scala-kod i REPL, som skript och som applikation.
\item Kunna implementera och köra igång ett Java-program. 
\item Känna till några grundläggande syntaxskillnader mellan Scala och Java, speciellt variabeldeklarationer och indexering i Array. 
\item Förstå vad ett block är.
\item Förstå vad en lokal variabel är.
\item Förstå hur nästlade block påverkar namnsynlighet och namnöverskuggning.
\item Förstå kopplingen mellan paketstruktur och klassfilstruktur.
\item Kunna skapa en jar-fil.
\item Kunna skapa dokumentation med scaladoc.
\end{Goals}

\begin{Preparations}
\item Studera teorin i kapitel~\ref{chapter:W02}.
\item Bekanta dig med grundläggande terminalkommandon; se appendix~\ref{appendix:terminal}. 
\item Bekanta dig med den editor du vill använda; se appendix~\ref{appendix:edit}.
\end{Preparations}

\BasicTasks %%%%%%%%%%%%%%%%

\Task  \emph{Datastrukturen \code+Range+.} Evaluera nedan uttryck i Scala REPL. Vad har respektive uttryck för värde och typ?

\Subtask \code{Range(1, 10)}

\Subtask \code{Range(1, 10).inclusive}

\Subtask \code{Range(0, 50, 5)}

\Subtask \code{Range(0, 50, 5).size}

\Subtask \code{Range(0, 50, 5).inclusive}

\Subtask \code{Range(0, 50, 5).inclusive.size}

\Subtask \code{0.until(10)}

\Subtask \code{0 until (10)}

\Subtask \code{0 until 10}

\Subtask \code{0.to(10)}

\Subtask \code{0 to 10}

\Subtask \code{0.until(50).by(5)}

\Subtask \code{0 to 50 by 5}

\Subtask \code{(0 to 50 by 5).size}

\Subtask \code{(1 to 1000).sum}


\Task \label{task:array} \emph{Datastrukturen \code+Array+.} Kör nedan kodrader i Scala REPL. Beskriv vad som händer.

\Subtask \code{val xs = Array("hej","på","dej", "!")}

\Subtask \code{xs(0)}

\Subtask \code{xs(3)}

\Subtask \code{xs(4)}

\Subtask \code{xs(1) + " " + xs(2)}

\Subtask \code{xs.mkString}

\Subtask \code{xs.mkString(" ")}

\Subtask \code{xs.mkString("(", ",", ")")}

\Subtask \code{xs.mkString("Array(", ", ", ")")}

\Subtask \code{xs(0) = 42}

\Subtask \code{xs(0) = "42"; println(x(0))}

\Subtask \code{val ys = Array(42, 7, 3, 8)}

\Subtask \code{ys.sum}

\Subtask \code{ys.min}

\Subtask \code{ys.max}

\Subtask \code{val zs = Array.fill(10)(42)}

\Subtask \code{zs.sum}

\Task \emph{Datastrukturen \code+Vector+.} Kör nedan kodrader i Scala REPL. Beskriv vad som händer.

\Subtask \code{val words = Vector("hej","på","dej", "!")}

\Subtask \code{words(0)}

\Subtask \code{words(3)}

\Subtask \code{words.mkString}

\Subtask \code{words.mkString(" ")}

\Subtask \code{words.mkString("(", ",", ")")}

\Subtask \code{words.mkString("Ord(", ", ", ")")}

\Subtask \code{words(0) = "42"}

\Subtask \code{val numbers = Vector(42, 7, 3, 8)}

\Subtask \code{numbers.sum}

\Subtask \code{numbers.min}

\Subtask \code{numbers.max}

\Subtask \code{val moreNumbers = Vector.fill(10000)(42)}

\Subtask \code{moreNumbers.sum}

\Subtask\Pen Jämför med uppgift \ref{task:array}. Vad kan man göra med en \code{Array} som man inte kan göra med en \code{Vector}?

\Task \emph{\code+for+-uttryck}. Evaluera nedan uttryck i Scala REPL. Vad har respektive uttryck för värde och typ?

\Subtask \code{for (i <- Range(1,10)) yield i}

\Subtask \code{for (i <- 1 until 10) yield i}

\Subtask \code{for (i <- 1 until 10) yield i + 1}

\Subtask \code{for (i <- Range(1,10).inclusive) yield i}

\Subtask \code{for (i <- 1 to 10) yield i}

\Subtask \code{for (i <- 1 to 10) yield i + 1}

\Subtask \code{(for (i <- 1 to 10) yield i + 1).sum}

\Subtask \code{for (x <- 0.0 to 2 * math.Pi by math.Pi/4) yield math.sin(x)}


\Task \emph{Metoden \code+map+ på en samling.} Evaluera nedan uttryck i Scala REPL. Vad har respektive uttryck för värde och typ?

\Subtask \code{Range(0,10).map(i => i + 1)}

\Subtask \code{(0 until 10).map(i => i + 1)}

\Subtask \code{(1 to 10).map(i => i * 2)}

\Subtask \code{(1 to 10).map(_ * 2)}

\Subtask \code{Vector.fill(10000)(42).map(_ + 43)}

\Task \emph{Metoden \code+foreach+ på en samling.} Kör nedan satser i Scala REPL. Vad händer?

\Subtask \code{Range(0,10).foreach(i => println(i))}

\Subtask \code{(0 until 10).foreach(i => println(i))}

\Subtask \code|(1 to 10).foreach{i => print("hej"); println(i * 2)}|

\Subtask \code{(1 to 10).foreach(println)}

\Subtask \code{Vector.fill(10000)(math.random).foreach(r => if (r > 0.99) print("pang!"))}


\Task \emph{Algoritm: SWAP.}

\Subtask Skriv med \emph{pseudo-kod} algoritmen SWAP. Beskriv på vanlig svenska, steg för steg, hur en variabel $temp$ används för mellanlagring vid värdebytet: 

\emph{Indata:} två heltalsvariabler $x$ och $y$ 

\emph{???}

\emph{Utdata:} variablerna $x$ och $y$ vars värden har bytt plats.

\Subtask Implementerar algoritmen SWAP. Ersätt \code{???} nedan med satser separerade av semikolon:

\begin{REPL}
scala> var (x, y) = (42, 43)
scala> ???
scala> println("x är " + x + ", y är " + y)
x är 43, y är 42
\end{REPL}



\Task \emph{Skript.} Skapa med hjälp av en editor en fil med namn \texttt{hello-script.scala} som innehåller denna enda rad:
\begin{Code}
println("hej skript")
\end{Code}
Spara filen och kör kommandot \code{scala hello-script.scala} i terminalen:
\begin{REPLnonum}
> scala hello-script.scala
\end{REPLnonum}

\Subtask Vad händer?

\Subtask Ändra i filen så att högerparentesen saknas. Spara och kör skriptfilen igen. Vad händer?

\Subtask Lägg till en sats sist i skriptet som skriver ut summan av de ett tusen stycken heltalen från och med 2 till och med 1001, så som visas nedan.
\begin{REPL}
> scala hello-script.scala
hej skript
501500
\end{REPL}

\Subtask Ändra i hello-script.scala genom att införa \code{val n = args(0).toInt} och använd \code{n} som övre gräns för summeringen av de n första heltalen.
\begin{REPL}
> scala hello-script.scala 5001
hej skript
12507501
\end{REPL}

\Subtask Vad blir det för felmeddelande om du glömmer ge programmet ett argument?


\Task \emph{Applikation med \code+main+-metod.} Skapa med hjälp av en editor en fil med namn \texttt{hello-app.scala}.
\begin{REPLnonum}
> gedit hello-app.scala
\end{REPLnonum}
Skriv dessa rader i filen:


\scalainputlisting{examples/hello-app.scala}

\Subtask Kompilera med \code{scalac hello-app.scala} och kör koden med \code{scala Hello}.
\begin{REPLnonum}
> scalac hello-app.scala
> ls
> scala Hello
\end{REPLnonum}
Vad heter filerna som kompilatorn skapar?

\Subtask Ändra i din kod så att kompilatorn ger följande felmeddelande: \\
\texttt{Missing closing brace}

\Subtask\Pen Varför behövs \code{main}-metoden?

\Subtask\Pen Vilket alternativ går snabbast att köra igång, ett skript eller en kompilerad applikation? Varför? Vilket alternativ kör snabbast när väl exekveringen är igång?


\Task \label{task:java} \emph{Java-applikation.} Skapa med hjälp av en editor en fil med namn \texttt{Hi.java}.
\begin{REPLnonum}
> gedit Hi.java
\end{REPLnonum}
Skriv dessa rader i filen:

\javainputlisting{examples/Hi.java}

\noindent Kompilera med \code{javac Hi.java} och kör koden med \code{java Hi}.
\begin{REPLnonum}
> javac Hi.java
> ls
> java Hi
\end{REPLnonum}

\Subtask\Pen Vad heter filen som kompilatorn skapat?

\Subtask\Pen Jämför signaturen för Java-programmets main-metod med signaturen för Scala-programmets main-metod. De betyder samma sak men syntaxen är olika. Beskriv skillnader och likheter i syntaxen.

\Subtask\Pen Vad blir det för felmeddelande om källkodsfilen och klassnamnet inte överensstämmer i ett Java-program?


\Task \emph{Algoritm: SUMBUG}. Nedan återfinns pseudo-koden för SUMBUG. 

\begin{algorithm}[H]
 \SetKwInOut{Input}{Indata}\SetKwInOut{Output}{Resultat}
 
 \Input{heltalet $n$}
 \Output{utskrift av summan av de första $n$ heltalen }
 $sum \leftarrow 0$ \\
 $i \leftarrow 1$  \\
 \While{$i \leq n$}{
  $sum \leftarrow sum + 1$
 }
 skriv ut $sum$
\end{algorithm}

\Subtask\Pen Kör algoritmen steg för steg med penna och papper, där du skriver upp hur värdena fär respektive variabel ändras. Det finns en bugg i algoritmen. Vilken? Rätta buggen.

\Subtask Skapa med hjälp av en editor filen \code{sumn.scala}. Implementera algoritmen SUM enligt den rättade pseudokoden och placera implementationen i en main-metod i ett objekt med namnet \code{sumn}. Du kan skapa indata \code{n} till algoritmen med denna deklaration i början av din main-metod: \\ \code{val n = args(0).toInt} \\ Vad ger applikationen för utskrift om du kör den med argumentet 8888? 

\begin{REPLnonum}
> scalac sumn.scala
> scala sumn 8888
\end{REPLnonum}

\Subtask Kontrollera att din implementation räknar rätt genom att jämföra svaret med detta uttrycks värde, evaluerat i Scala REPL:
\begin{REPLnonum}
scala> (1 to 8888).sum
\end{REPLnonum}

\Subtask Implementera algoritmen SUM enligt pseudokoden ovan, men nu i Java. Skapa filen \code{SumN.java} och använd koden från uppgift \ref{task:java} som mall för att deklarera den publika klassen \code{SumN} med en main-metod. Några tips om Java-syntax och standarfunktioner i Java: 

\begin{itemize}[noitemsep, nolistsep]
\item Alla satser i Java måste avslutas med semikolon.
\item Heltalsvariabler deklareras med nyckelordet \lstinline[language=Java]{int} (litet i). 
\item Typnamnet ska stå \emph{före} namnet på variabeln. Exempel: \\ \lstinline[language=Java]{int sum = 0;}
\item Indexering i en array görs i Java med hakparenteser: \code{args[0]}
\item I stället för Scala-uttrycket \code{args(0).toInt}, använd Java-uttrycket: \\ \code{Integer.parseInt(args[0])}
\item \code{while}-satser i Scala och Java har samma syntax.
\item Utskrift i Java görs med \code{System.out.println}
\end{itemize}


\Task \emph{Algoritm: MAXBUG}. Nedan återfinns pseudo-koden för MAXBUG. 

\begin{algorithm}[H]
 \SetKwInOut{Input}{Indata}\SetKwInOut{Output}{Resultat}
 
 \Input{Array $args$ med strängar som alla innehåller heltal}
 \Output{utskrift av största heltalet }
 $max \leftarrow$ det minsta heltalet som kan uppkomma  \\
 $n \leftarrow $ antalet heltal \\
 $i \leftarrow 0$ \\
 \While{$i < n$}{
   $x \leftarrow args(i).toInt$ \\
   \If{( x > $max$)}{$max \leftarrow x$} 
  % $i \leftarrow i + 1$
 }
 skriv ut $max$
\end{algorithm}

\Subtask\Pen Kör med penna och papper. Det finns en bugg i algoritmen ovan. Vilken? Rätta buggen.

\Subtask Implementera algoritmen MAX (utan bugg) som en Scala-applikation. Tips:
\begin{itemize}[noitemsep, nolistsep]
\item Det minsta heltalet som någonsin kan uppkomma: \code{Int.MinValue}
\item Antalet element i $args$ ges av: \code{args.size}
\end{itemize}

\begin{REPL}
> gedit maxn.scala
> scalac maxn.scala
> scala maxn 7 42 1 -5 9
42
\end{REPL}

\Subtask\Pen \label{subtask:arg0} Skriv om algoritmen så att variablen $max$ initialiseras med det första talet i sekvensen. 

\Subtask Implementera den nya algoritmvarianten från uppgift \ref{subtask:arg0} och prova programmet. Vad händer om $args$ är tom?

\Task \emph{Block, namnsynlighet, namnöverskuggning}. Kör nedan kod i Scala REPL eller i Kojo. Vad händer nedan? Varför?

\Subtask \code|val a = {1 + 1; 2 + 2; 3 + 3; 4 + 4}; println(a)|

\Subtask \code|val b = {1; 2; 3; {val b = 4; b + b; b + 1}}; println(b)|

\Subtask \code|{val a = 42; println(a)}|

\Subtask \code|{val a = 42}; println(a)|

\Subtask \code|{val a = 42; {val a = 43; println(a)}; println(a)}|

\Subtask \code|{var a = 42; {a = a + 1}; var a = 43}|

\Subtask \code|{var a = 42; {a = a + b; var b = 43}; println(a)}|

\Subtask \code|{var a = 42; {var b = 43; a = a + b}; println(a)}|

\Subtask \code|{var a = 42; {a = a + b; def b = 43}; println(a)}|

\Subtask \code|{object a{var b=42;object a{var a=43}};println(a.b+a.a.a)}|

\Subtask 

\begin{Code}
{
  object a {
    var b = 42
    object a {
      var a = 43
    }
  }
  println(a.b + a.a.a)
}
\end{Code}

\Subtask Vad är fördelen med att namn deklararerade inne i ett block är lokala i stället för globala?


\Task \label{task:package} \emph{Paket, \code{import} och klassfilstrukturer.} Med Java-8-plattformen kommer 4240 färdiga klasser, som är organiserade i 217 olika paket.\footnote{Se Stackoverflow: \href{http://stackoverflow.com/questions/3112882/how-many-classes-are-there-in-java-standard-edition}{how-many-classes-are-there-in-java-standard-edition}}

\Subtask Vilka paket finns i paketet javax som börjar på s?

\begin{REPLnonum}
scala> javax.s   //tryck på TAB-tangenten
\end{REPLnonum} 

\Subtask Kör raderna nedan i REPL. Beskriv vad som händer för varje rad.
\begin{REPL}[numbers=left, numberstyle=\color{black}\ttfamily\scriptsize\selectfont]
scala> import javax.swing.JOptionPane
scala> def msg(s: String) = JOptionPane.showMessageDialog(null, s)
scala> msg("Hej på dej!")
scala> def input(msg: String) = JOptionPane.showInputDialog(null, msg)
scala> input("Vad heter du?")
scala> import JOptionPane.{showOptionDialog => optDlg}
scala> def inputOption(msg: String, opt: Array[Object]) = 
         optDlg(null, msg, "Option", 0, 0, null, opt, opt(0))
scala> inputOption("Vad väljer du?", Array("Sten", "Sax", "Påse"))
\end{REPL} 

\Subtask\Pen Vad hade du behövt ändra på efterföljande rader om import-satsen på rad 1 ovan ej hade gjorts?

\Subtask Skapa med en editor filen paket.scala och kompilera. Rita en bild av hur katalogstrukturen ser ut.

\begin{Code}
package gurka.tomat.banan

package p1 {
  package p11 {
    object hello {
      def hello = println("Hej paket p1.p11!")
    }
  }
  package p12 {
    object hello {
      def hello = println("Hej paket p1.p12!")
    }
  }
}

package p2 {
  package p21 {
    object hello {
      def hello = println("Hej paket p2.p21!")
    }
  }
}

object Main {
  def main(args: Array[String]): Unit = {
    import p1._
    p11.hello.hello
    p12.hello.hello
    import p2.{p21 => apelsin}
    apelsin.hello.hello
  }
}
\end{Code}

\begin{REPL}
> gedit paket.scala
> scalac paket.scala
> scala gurka.tomat.banan.Main
> ls -R
\end{REPL}

\Task \emph{Skapa \code{jar}-filer och använda classpath}

\Subtask Skriv kommandot \code{jar} i terminalen och undersök vad som finns för optioner. Se speciellt ''Example 1.'' i hjälputskriften. Vilket kommando ska du använda för att packa ihop flera filer i en enda jar-fil?

\Subtask Som en fortsättning på uppgift \ref{task:package}, packa ihop biblioteket \code{gurka} i en jar-fil med nedan kommando, samt kör igång REPL med jar-filen på classpath.

\begin{REPL}
> jar cvf mittpaket.jar gurka
> scala -cp mittpaket.jar
scala> gurka.tomat.banan.Main.main(Array())
\end{REPL}

 
\Task \emph{Skapa dokumentation med \code{scaladoc}-kommandot}

\Subtask Som en fortsättning på uppgift \ref{task:package}, kör nedan kommando i terminalen:

\begin{REPL}
> scaladoc paket.scala
> ls
> firefox index.html   # eller öppna index.html i valfri webbläsare
\end{REPL}

Vad händer?

\Subtask Lägg till några fler metoder i något av objekten i filen \code{paket.scala} och lägg även till några dokumentationskommentarer. Kompilera om och kör. Generera om dokumentationen. 

\begin{verbatim}
//... ändra i filen paket.scala

/** min paketdokumentationskommentar p2 */
package p2 {
  /** min paketdokumentationskommentar p21 */
  package p21 {
    /** ett hälsningsobjekt */
    object hello {
      /** en hälsningsmetod i p2.p21 */
      def hello = println("Hej paket p2.p21!")
      
      /** en metod som skriver ut tiden */
      def date = println(new java.util.Date)
    }
  }
}

\end{verbatim}

\begin{REPL}
> gedit paket.scala
> scalac paket.scala
> jar cvf mittpaket.jar gurka
> scala -cp mittpaket.jar
scala> gurka.tomat.banan.p2.p21.hello.date
scala> :q
> scaladoc paket.scala
> firefox index.html
\end{REPL}

\ExtraTasks %%%%%%%%%%%%%%%%%%%



\AdvancedTasks %%%%%%%%%%%%%%%%%


\Task ArrayBuffer vs Vector vs Array och metoden append

\Task Läs om krullparenetser och vanliga parenteser på stack overflow: \\ \href{http://stackoverflow.com/questions/4386127/what-is-the-formal-difference-in-scala-between-braces-and-parentheses-and-when}{what-is-the-formal-difference-in-scala-between-braces-and-parentheses-and-when}

\Task Bygg vidare på koden nedan och gör ett Sten-Sax-Påse-spel\footnote{\href{https://sv.wikipedia.org/wiki/Sten,\_sax,\_p\%C3\%A5se}{sv.wikipedia.org/wiki/Sten,\_sax,\_p\%C3\%A5se}} som även meddelar vem som vinner. Koden fungerar att köra som den är, men funktionen \code{winnerMsg} är ej klar. \emph{Tips:} Du kan använda modulo-räkning med \%-operatorn för att avgöra vem som vinner.

\begin{Code}[basicstyle=\ttfamily\footnotesize\selectfont]]
object Rock {
  import javax.swing.JOptionPane
  import JOptionPane.{showOptionDialog => optDlg}
  
  def inputOption(msg: String, opt: Vector[String]) = 
    optDlg(null, msg, "Option", 0, 0, null, opt.toArray[Object], opt(0))
    
  def msg(s: String) = JOptionPane.showMessageDialog(null, s)  
  
  val opt =  Vector("Sten", "Sax", "Påse")
   
  def userChoice = inputOption("Vad väljer du?", opt)
  
  def computerChoice = (math.random * 3).toInt     
  
  def winnerMsg(user: Int, computer: Int) = "??? vann!"
  
  def main(args: Array[String]): Unit = {
    var keepPlaying = true
    while (keepPlaying) {
      val u = userChoice
      val c = computerChoice
      msg("Du valde " + opt(u) + "\n" + 
          "Datorn valde " + opt(c) + "\n" + 
          winnerMsg(u, c))
      if (u != c) keepPlaying = false 
    }
  }
}
\end{Code}
%!TEX encoding = UTF-8 Unicode
%!TEX root = ../compendium.tex

% INGEN LAB DENNA VECKA
%!TEX encoding = UTF-8 Unicode
%!TEX root = ../compendium.tex

\input{generated/w03-chaphead-generated.tex}



%!TEX encoding = UTF-8 Unicode
%!TEX root = ../compendium.tex

\Exercise{\ExeWeekTHREE}

\begin{Goals}
\item Kunna skapa och använda funktioner med en eller flera parametrar, default-argument, namngivna argument, och uppdelad parameterlista.
\item Kunna använda funktioner som äkta värden. 
\item Kunna applicera en funktion på element i en samling.
\item Förstå skillnader och likheter mellan en funktion och en procedur.
\item Förstå skillnader och likheter mellan en värde-anrop och namnanrop.
\item Kunna skapa en procedur i form av en enkel kontrollstruktur med fördröjd evaluering av ett block.
\item Kunna skapa och använda objekt som moduler.
\item Förstå skillnaden mellan äkta funktioner och funktioner med sidoeffekter.
\item Kunna skapa och använda variabler med fördröjd initialisering och förstå när de är användbara.
\item Kunna förklara hur nästlade funktionsanrop fungerar med hjälp av begreppet aktiveringspost.
\item Kunna skapa och använda lokala funktioner, samt förstå nyttan med lokala funktioner.
\item Känna till att funktioner är objekt med en \code{apply}-metod.
\item Känna till stegade funktioner och kunna använda partiellt applicerade argument. 
\item Känna till rekursion och kunna förklara hur rekursiva funktioner fungerar med hjälp av   anropsstacken.
\item Känna till svansrekursion och att svansrekursiva funktioner kan optimeras till loopar.
\end{Goals}

\begin{Preparations}
\item Studera teorin i kapitel~\ref{chapter:W03}.
\end{Preparations}

\BasicTasks %%%%%%%%%%%%%%%%

\Task \label{task:funcall}\emph{Definiera och anropa funktioner.} En funktion med två parametrar definieras med följande syntax i Scala: \vspace{0.5em} \\  \texttt{\code{def} \textit{namn}(\textit{parameter1}: \textit{Typ1}, \textit{parameter2}: \textit{Typ2}): \textit{Returtyp} = \textit{returvärde}}

\Subtask Definiera en funktion med namnet \code{öka} som har en heltalsparameter \code{x} och som returnerar \code{x + 1}. Ange returtypen explicit. Testa funktionen i REPL med argumentet 42.

\begin{REPL}
scala> ???  // definiera funktionen öka
scala> öka(42)
43
\end{REPL}

\Subtask\Pen Vad har funktionen \code{öka} i föregående uppgift för returtyp?

\Subtask\Pen Vad gör kompilatorn om du utelämnar returtypen?

\Subtask\Pen Varför kan det vara bra att ange returtypen explicit?

\Subtask\Pen Vad är det för skillnad mellan parameter och argument?
 
\Subtask Vad har uttrycket \code{öka(öka(öka(öka(42))))} för värde?

\Subtask Definera funktionen \code{minska(x: Int): Int} med returvärdet \code{x - 1}.

\Subtask Vad är värdet av uttrycket \code{öka(minska(öka(öka(minska(minska(42))))))}


\Task \emph{Funktion med flera parametrar.} Definiera i REPL två funktioner \code{sum} och \code{diff} med två heltalsparametrar som returnerar summan respektive differensen av argumenten: \\
\code{def sum(x: Int, y: Int): Int = x + y} \\
\code{def diff(x: Int, y: Int): Int = x - y} \\
Vad har nedan uttryck för värden? Förklara vad som händer.

\Subtask \code{diff(0, 100)}

\Subtask \code{diff(100, add(42, 43))}

\Subtask \code{sum(sum(42, 43), diff(100, sum(0, 0))}

\Subtask \code{sum(diff(Byte.MaxValue, Byte.MinValue),1)}

\Task \emph{Funktion med default-argument.} Förklara vad som händer här?
\begin{REPL}
scala> def inc(i: Int, j: Int = 1) = i + j
scala> inc(42, 2)
scala> inc(42, 1)
scaka> inc(42)
\end{REPL}

\Task \emph{Funktionsanrop med namngivna argument.} 
\begin{REPL}
scala> def skrivNamn(förnamn: String, efternamn: String) = 
         println("Namn: " + efternamn + ", " + förnamn)
scala> skrivNamn("Kim", "Robinson")
scala> skrivNamn(förnamn = "Viktor", efternamn = "Oval")
scaka> skrivNamn(efternamn = "Triangelsson", förnamn = "Stina")
\end{REPL}

\Subtask Förklara vad som händer ovan?

\Subtask\Pen Vad är fördelen med namngivna argument?



\Task \emph{Applicera en funktion på elementen i en samling.} Använd dina funktioner \code{öka} och \code{minska} från uppgift \ref{task:funcall}. Vad har nedan uttryck för värde? Förklara vad som händer.

\Subtask \code{for (i <- 0 to 4) yield öka(i)}

\Subtask \code{for (i <- 1 to 5) yield minska(i)}

\Subtask \code{(0 to 4).map(i => öka(i))}

\Subtask \code{(1 to 5).map(i => minska(i))}

\Subtask \code{(0 to 4).map(öka)}

\Subtask \code{(1 to 5).map(minska)}

\Subtask \code{Vector(12, 3, 41, -8).map(öka)}

\Subtask \code{Vector(12, 3, 41, -8).map(öka).map(minska).map(minska)}



\Task En funktion som inte returnerar något intressant värde, men som anropas för det den \emph{gör} kallas \textbf{procedur}. Definiera följande procedur i REPL: \\ 
\code{def tUvirks(msg: String) = println(msg.reverse)} \\
Vad skriver nedan satser ut? Förklara vad som händer.

\Subtask \code{println("sallad".reverse)}

\Subtask \code{tUvirks("sallad")}

\Subtask \code{val x = tUvirks("sallad"); println(x)}

\Subtask \code{def enhetsvärdet = (); println(enhetsvärdet)}

\Subtask \code{def bortkastad: Unit = 1 + 1; println(bortkastad)}

\Subtask \code|def bortkastad2 = {val x = 1 + 1}; println(bortkastad2)|

\Subtask\Pen Varför är det bra att explicit ange \code{Unit} som returtyp för procedurer?


\Task \emph{Värdeanrop och namnanrop (fördröjd evaluering, ''lata'' argument).} Deklarera nedan funktioner i REPL eller Kojo.

\begin{Code}
def snark: Int = {print("snark "); Thread.sleep(1000); 42}
def callByValue(x: Int) = x + x
def callByName(x: => Int) = x + x
\end{Code}

Evaluera nedan uttryck. Förklara vad som händer.

\Subtask \code{snark}

\Subtask \code{snark; snark; snark}

\Subtask \code{callByValue(1)}

\Subtask \code{callByName(1)}

\Subtask \code{callByValue(snark)}

\Subtask \code{callByName(snark)}


\Subtask Förklara vad som händer här:
\begin{REPL}
scala> def görDetta(block: => Unit) = block
scala> görDetta(println("hej"))
scala> görDetta{println("goddag")}
scala> görDetta{println("hej"); println("svejs")}
scala> def görDettaTvåGånger(block: => Unit) = {block; block}
scala> görDettaTvåGånger{println("goddag")}
\end{REPL}


\Task \emph{Uppdelad parameterlista.} Man kan dela upp parametrarna till en funktion i flera parameterlistor. Förklara vad som händer här:
\begin{REPL}
scala> def add(a: Int)(b: Int) = a + b
scala> add(22)(20)
scala> add(22)(add(1)(19))
\end{REPL}


\Task \emph{Skapa din egen kontrollstruktur.} Använd fördröjd evaluering och stegad funktion och skapa din egen loop-konstruktion.
\begin{REPL}
scala> def upprepa(n: Int)(block: => Unit) = {
         var i = 0
         while (i < n) {block; i = i + 1}
       }
scala> upprepa(10)(println("hej"))
scala> upprepa(1000){
  val tärning = (math.random * 6 + 1).toInt
  print(tärning + " ")
}
\end{REPL}
Förklara vad som händer ovan. (Det är så som \code{upprepa} i Kojo är definierad.)


\Task \emph{Funktion som värde.} Funktioner är äkta värden i Scala.

\Subtask \label{subtask:funcval} Förklara vad som händer nedan. Notera understrecket på rad 4:

\begin{REPL}[numbers=left, numberstyle=\color{black}\ttfamily\scriptsize\selectfont]
scala> def inc(x: Int): Int = x + 1
scala> inc(42)
scala> Vector(12, 3, 41, -8).map(inc)
scala> val f = inc _
scala> Vector(12, 3, 41, -8).map(f)
\end{REPL}

\Subtask Vad händer om du bara skriver \code{val f = inc} utan understreck?

\Subtask På liknande sätt som i uppgift \ref{subtask:funcval}: definiera en funktion \code{dec} som i stället \emph{minskar} med 1. Deklarera ett funktionsvärde \code{g} som tilldelas funktionen \code{dec} och kör sedan \code{g} på varje element i \code{Vector(12, 3, 41, -8)} med metoden \code{map}.

\Subtask\Pen Vad har variablerna \code{f} och \code{g} ovan för typ?

\Subtask Förklara vad som händer nedan. Vad får \code{d} och \code{h} för värde?

\begin{REPL}
scala> def räkna(x: Int, f: Int => Int) = f(x,y)
scala> def dubbla(x: Int) = 2 * x
scala> def halva(x: Int) = x / 2
scala> val d = räkna(42, dubbla)
scala> val h = räkna(42, halva)
\end{REPL}

\Task\emph{Stegade funktioner (''Curry-funktioner'').} Förklara vad som händer nedan.
\begin{REPL}
scala> def sum(a: Int)(b: Int) = a + b
scala> sum(1)(2)
scala> val f = sum(42) _
scala> f(1)
scala> val inc = sum(1) _
scala> val dec = sum(-1) _
scala> inc(42)
scala> dec(42)
\end{REPL}

\Task \emph{Objekt som moduler.} 

\Subtask Lär dig följande terminologi utantill: 

\begin{itemize}[noitemsep, nolistsep]
\item Ett objekt som samlar funktioner och variabler kallas även en \textbf{modul}. 
\item Funktioner i objekt kallas även \textbf{metoder}. 
\item Variabler och metoder i objekt kallas \textbf{medlemmar}. 
\item Moduler kan i sin tur innehålla moduler, i godtyckligt nästlingsdjup. 
\item Man kommer åt innehållet i en modul med \textbf{punktnotation}. 
\item Med \textbf{import} slipper man punktnotation. 
\item Ett objekt med variabler sägs ha ett \textbf{tillstånd}.
\end{itemize}

\Subtask Deklarera modulerna \code{stringstat} och \code{Test} nedan i REPL eller i Kojo. 

\begin{Code}
object stringstat {
  object stringfun {
    def sentences(s: String): Array[String] = s.split('.')
    def words(s: String): Array[String] = s.split(' ')
    def countWords(s: String): Int = words(s).size
    def countSentences(s: String): Int = sentences(s).size
  }
  
  object statistics {
    var history = ""
    def printFreq(s: String): Unit = {
      println("\n---- Frekvenser ----")
      println("Antal tecken:   " + s.size)
      println("Antal ord:      " + stringfun.countWords(s))
      println("Antal meningar: " + stringfun.countSentences(s))
      history = history + " " + s
    }
    def printTotal: Unit = printFreq(history)
  }
}
  
object Test {
  import stringstat._
  def apply(n: Int = 42): Unit = {
    val s1 = "Fem myror är fler än fyra elefanter. Ät gurka."
    val s2 = "Galaxer i mina braxer. Tomat är gott. Hejsan."
    statistics.printFreq(s1 * n)
    statistics.printFreq(s2 * n)
    statistics.printTotal
  }
}
\end{Code}

\Subtask Anropa \code{Test()} och förklara vad som händer. Vad skrivs ut?

\Subtask Vilket av objekten i modulen \code{stringstat} har tillstånd och vilket av objekten är tillståndslöst? Vad består tillståndet av?


\Task \emph{Äkta funktioner.} En \textbf{äkta funktion} ger alltid samma resultat med samma argument.  

\begin{Code}
object inSearchOfPurity {
  var x = 0
  val y = x
  def inc(i: Int) = i + 1
  def oink(i: Int) = {x = x + i; "Pig says oink " + x}
  def addX(i: Int): Int = x + i
  def addY(i: Int): Int = y + i
  def isPalindrome(s: String): Boolean = s == s.reverse
  def rnd(min: Int, max: Int) = math.random * max + min
}
\end{Code}

\Subtask\Pen Vilka funktioner i objektet \code{inSearchOfPurity} är äkta funktioner?

\Subtask \label{subtask:nonpure} Anropa de funktioner som inte är äkta i REPL och demonstrera med exempel att de kan ge olika resultat för samma argument.

\Subtask Vad objektets är tillstånd efter dina körningar i uppgift \ref{subtask:nonpure}? 

\Subtask Vilken del av tillståndet i objektet är oföränderligt?



\Task Funktioner är objekt med en \code{apply}-metod. 

\Subtask Förklara vad som händer här:
\begin{REPL}
scala> object plus { def apply(x: Int, y: Int) = x + y }
scala> plus.apply(42,43)
scala> plus(42, 43)
scala> val add: (Int, Int) => Int = (x, y) => x + y
scala> add(42, 42)
scala> add.   // tryck på TAB
scala> add.apply(42, 42)
scala> val inc = add.curried(1)
scala> inc(42)
\end{REPL}

\Subtask Definiera i REPL ett objekt som heter \code{Slumptal} som har en \code{apply}-metod som tar två heltalsparametrar \code{a} och \code{b} och som med hjälp av \code{math.random} returnerar ett slumpal i intervallet $[a, b]$. Anropa objektets \code{apply}-metod med \code{(1 to 100).foreach(println(???))}, där \code{???} ersätts först med punktnotation och sedan med funktionsappliceringssyntax. 



\Task \emph{Fördröjd initialisering (''lata'' variabler).} 

\Subtask \label{subtask:delayalloc} Förklara vad som händer här:
\begin{REPL}
scala> val olat = 42
scala> lazy val lat = 42
scala> println(lat)
scala> val nu = {Thread.sleep(1000); println("nu"); 42}
scala> lazy val sen = {Thread.sleep(1000); println("sen"); 42}
scala> def igen = {Thread.sleep(1000); println("hver gang"); 42}
scala> println(nu)
scala> println(sen)
scala> println(igen)
scala> println(nu)
scala> println(sen)
scala> println(igen)
scala> object m {lazy val stor = Array.fill(1e9.toInt)(liten); val liten = 42}
scala> m.liten
scala> m.stor
\end{REPL}

\Subtask\Pen Vad är skillnaden mellan \code{val}, \code{lazy val} och \code{def}, vad gäller \emph{när} evalueringen sker?


\Subtask \label{subtask:forwardref} Förklara vad som händer här:
\begin{REPL}
scala> object objektÄrLata { val sen = { println("nu!"); 42 } }
scala> objektÄrLata
scala> objektÄrLata.sen
scala> {val x = y; val y = 42}
scala> object buggig {val a = b; val b = 42}
scala> buggig.a
scala> object funkar {lazy val a = b; val b = 42}
scala> funkar.a
scala> object nowarning {val many = Array.fill(10)(one); val one = 1}
scala> nowarning.many
\end{REPL}

\Subtask\Pen Med ledning av uppgift \ref{subtask:delayalloc} och uppgift \ref{subtask:forwardref}, beskriv två olika situationer när kan man ha nytta av \code{lazy val}?


\Task \emph{Aktiveringspost.} Antag att vi bara kan addera eller subtrahera med ett. Då kan man ändå skapa en additionsfunktion på nedan (ganska omständliga) sätt. Skriv nedan program i en editor, kompilera och exekvera. 
\begin{Code}
object Count {
  def inc(x: Int) = {println("inc[x = " + x + "]"); x + 1}
  def dec(x: Int) = {println("dec[x = " + x + "]"); x - 1}

  def add(x: Int, y: Int) = {
    println("add[x = " + x + ", y = " + y + "]")
    var result = x; 
    var i = 0; 
    while (i < math.abs(y)){
      result = if (y >= 0) inc(result) else dec(result)
      i = i + 1
    }
    result
  }

  def main(args: Array[String]): Unit = {
    val x =  inc(dec(inc(0)))
    println(x)
    val y = add(1, add(1, add(1, -2)))
    println(y)
  }
}
\end{Code}

\Subtask Vad skrivs ut? Förklara vad som händer.

\Subtask\Pen Rita hur anropsstacken förändras under exekveringen av main-metoden.


\Task \emph{Lokala funktioner.} Skapa nedan program i en editor, kompilera och exekvera. I programmet nedan har metoden \code{add} två lokala funktioner som skiljer sig från metoderna med samma namn. 
\begin{Code}
object Count {
  def inc(x: Int) = x + 1
  def dec(x: Int) = x - 1

  def add(x: Int, y: Int) = {
    def inc(x: Int) = {println("inc[x = " + x + "]"); x + 1}
    def dec(x: Int) = {println("dec[x = " + x + "]"); x - 1}
    println("add[x = " + x + ", y = " + y + "]")
    var result = x; 
    var i = 0; 
    while (i < math.abs(y)){
      result = if (y >= 0) inc(result) else dec(result)
      i = i + 1
    }
    result
  }

  def main(args: Array[String]): Unit = {
    val x =  inc(dec(inc(0)))
    println(x)
    val y = add(1, add(1, add(1, -2)))
    println(y)
  }
}
\end{Code}

\Subtask Vad skrivs ut? Förklara vad som händer.

\Subtask\Pen Vilka fördelar finns med lokala funktioner?



\Task \emph{Rekursion.} 

\Subtask Förklara vad som händer nedan. 

\begin{REPL}
scala> def countdown(x: Int): Unit = if (x > 0) {println(x); countdown(x -1)}
scala> countdown(10)
scala> countdown(-1)
scala> def finalCountdown(x: Byte): Unit = 
         {println(x); Thread.sleep(100); finalCountdown((x-1).toByte); 1 / x}
scala> finalCountdown(Byte.MaxValue)
\end{REPL}

\Subtask Vad händer om du gör satsen som riskerar division med noll \emph{före} det rekursiva anropet i funktionen \code{finalCountdown} ovan?

\Subtask Förklara vad som händer nedan. Varför tar sista raden längre tid än näst sista raden?
\begin{REPL}
scala> def signum(a: Int): Int = if (a >= 0) 1 else -1 
scala> def add(x: Int, y: Int): Int = 
         if (y == 0) x else add(x + 1, y - signum(y))
scala> add(100,100)
scala> add(Int.MaxValue, 0)
scala> add(0, Int.MaxValue)
\end{REPL}



\ExtraTasks %%%%%%%%%%%%%%%%%%%

\Task Visa anropsstacken genom att kasta undantag.

\AdvancedTasks %%%%%%%%%%%%%%%%%



\Task \emph{Kolla bajtkoden.}
\begin{REPL}
scala> def plusxy(x: Int, y: Int) = x + y
scala> :javap plusxy
\end{REPL}

\Subtask Leta upp raden \code{public int plusxy(int, int);} och studera koden efter \code{Code:} och försök gissa vilken instruktion som utför själva additionen.

\Subtask Lägg till en parameter till: \\ \code{def plusxyz(x: Int, y: Int, z: Int) = x + y + z}
\\ och studera bajtkoden med \code{:javap plusxyz}. Vad skiljer bajtkoden mellan \code{plusxy} och \code{plusxyz}?

\Subtask\Pen Läs om bajtkod här: \href{https://en.wikipedia.org/wiki/Java\_bytecode}{en.wikipedia.org/wiki/Java\_bytecode}. Vad betyder den inledande bokstaven i additionsinstruktionen?


\Task \emph{Undersök svansrekursion genom att kasta undantag.} Förklara vad som händer. Kan du hitta bevis för att kompilatorn kan optimera rekursionen till en vanlig loop?

\begin{REPL}
scala> def explode = throw new Exception("BANG!!!")
scala> explode
scala> lastException.printStackTrace
scala> def countdown(n: Int): Unit = 
         if (n == 0) explode else countdown(n-1)
scala> countdown(10)
scala> lastException.printStackTrace
scala> def countdown2(n: Int): Unit = 
         if (n == 0) explode else {countdown2(n-1); print("no tailrec")}
scala> countdown2(10)
scala> countdown2(1000)
scala> lastException
scala> lastException.getStackTrace.size
scala> :javap countdown
scala> :javap countdown2
\end{REPL}

\Task \emph{\code{@tailrec}-annotering.} Du kan be kompilatorn att ge felmeddelande om den inte kan optimera koden till en loop och därmed öka prestanda och unvika en överfull anropssstack \Eng{stack overflow}. Prova nedan rader i REPL och förklara vad som händer. 
\begin{REPL}
scala> def countNoTailrec(n: Long): Unit = 
         if (n <= 0L) println("Klar! " + n) else {countNoTailrec(n-1L); ()}
scala> countNoTailrec(1000L)
scala> countNoTailrec(100000L) 
scala> import scala.annotation.tailrec
scala> @tailrec def countNoTailrec(n: Long): Unit = 
         if (n <= 0L) println("Klar! " + n) else {countNoTailrec(n-1L); ()}        
scala> @tailrec def countTailrec(n: Long): Unit = 
         if (n <= 0L) println("Klar! " + n) else countTailrec(n-1L)
scala> countTailrec(1000L)
scala> countTailrec(100000L)
scala> countTailrec(Int.MaxValue.toLong * 2L)
\end{REPL}



%!TEX encoding = UTF-8 Unicode
%!TEX root = ../compendium.tex

\Lab{\LabWeekTHREE}

\begin{Goals}
\item Att lära sig.
\end{Goals}

\begin{Preparations}
\item Att göra.
\end{Preparations}

\subsection{Obligatoriska uppgifter}

\Task En labbuppgiftsbeskrivning.

\Subtask En underuppgift.

\Subtask En underuppgift.

\subsection{Frivilliga extrauppgifter}

\Task En labbuppgiftsbeskrivning.

\Subtask En underuppgift.

\Subtask En underuppgift.
%!TEX encoding = UTF-8 Unicode

%!TEX root = ../compendium.tex

\input{generated/w04-chaphead-generated.tex}
\clearpage

\input{../slides/body/lect-week04-datastruct.tex}
    
%!TEX encoding = UTF-8 Unicode
%!TEX root = ../compendium.tex

\Exercise{\ExeWeekFOUR}

\begin{Goals}
\item Kunna skapa och använda tupler, som variabelvärden, parametrar och returvärden.

\item Förstå skillnaden mellan ett objekt och en klass och kunna förklara betydelsen av begreppet instans.

\item Kunna skapa och använda attribut som medlemmar i objekt och klasser och som som klassparametrar.

\item Beskriva innebörden av och syftet med att ett attribut är privat.

\item Kunna byta ut implementationen av metoden \code{toString}.

\item Kunna skapa och använda en objektfabrik med metoden \code{apply}.

\item Kunna skapa och använda en enkel case-klass.

\item Kunna använda operatornotation och förklara relationen till punktnotation.

\item Förstå konsekvensen av uppdatering av föränderlig data i samband med multipla referenser.

\item Känna till och kunna använda några grundläggande metoder på samlingar.

\item Känna till den principiella skillnaden mellan \code{List} och \code{Vector}.

\item Kunna skapa och använda en oföränderlig mängd med klassen \code{Set}.

\item Förstå skillnaden mellan en mängd och en sekvens.

\item Kunna skapa och använda en nyckel-värde-tabell, \code{Map}.

\item Förstå likheter och skillnader mellan en \code{Map} och en \code{Vektor}.
\end{Goals}

\begin{Preparations}
\item Studera teorin i kapitel~\ref{chapter:W04}.
\end{Preparations}

\BasicTasks %%%%%%%%%%%%%%%%

\Task \emph{En enkel datastruktur: tupel.} Du kan samla olika data i en tupel. Du kommer åt värdena med en metod som har namnet understreck följt av ordningsnumret.
\begin{REPL}
scala> val namn = ("Pippi", "Långstrump")
scala> namn._1
scala> namn._2
scala> println("Förnamn: " + namn._1 + "\nEfternamn:" + namn._2)
\end{REPL}

\Subtask Definiera en oföränderlig variabel med namnet \code{pt} som representerar en punkt med x-koordinaten 15.9 och y-koordinaten 28.9. Använd sedan \code{math.hypt} för att ta reda på avståndet från origo till punkten. Vad blir svaret?

\Subtask Du kan dela upp en tupel i sina beståndsdelar så här:
\begin{REPLnonum}
scala> val (förnamn, efternamn) = ("Ronja", "Rövardotter")
\end{REPLnonum}
Dela upp din punkt \code{pt} i sina beståndsdelar och kalla delarna \code{x} och \code{y}

\Subtask Värdena i en tupel kan ha olika typ. 
\begin{REPLnonum}
scala> val creature = ("Doktor", "Krokodil", 65.0, false)
scala> val (title, name, weight, isHuman)  = creature
\end{REPLnonum}
Vilken typ har 4-tupeln \code{creature} ovan?

\Subtask \label{subtask:tuplecoll} Tupler kan ingå i samlingar.
\begin{REPLnonum}
scala> val pts = Vector((0.0, 0.0), (1.0, 0.0), (1.0, 1.0), (0.0, 1.0)) 
scala> pts.foreach(println)
\end{REPLnonum}
Vilken typ har vektorn \code{pts} ovan?


\Subtask För 2-tupler finns ett kortare skrivsätt:
\begin{REPLnonum}
scala> ("Skåne", "Malmö")
scala> "Skåne -> "Malmö"
scala> val huvudstäder = Vector("Sverige" -> "Stockholm", "Norge" -> "Oslo") 
\end{REPLnonum}
Lägg till fler huvudstäder i vektorn ovan.

\Subtask Funktioner kan ta tupler som parametrar.
\begin{REPL}
scala> def length(pt: (Double, Double)) = math.hypot(pt._1, pt._2) 
scala> length((3.0, 4.0))
scala> length(3.0, 4.0)    //kompilatorn lägger till parenteserna innan anrop
\end{REPL}
Applicera funktionen \code{length} ovan på alla tupler i samlingen \code{pts} från uppgift \ref{subtask:tuplecoll} med \code{map}. Vad får resultatet för värde och typ?

\Subtask Funktioner kan ge tupler som resultat.
\begin{REPL}
scala> def div(a: Int, b: Int) = (a / b, a % b)
scala> div(10, 3)
scala> (div(9,2), div(10,2))
scala> (div(9,2)._2, div(10,2)._2)
scala> val nOdd = (1 to 10).map(i => div(i, 2)._2).sum
\end{REPL}
Förklara vad som händer ovan. Använd \code{div} ovan för att ta reda på hur många udda tal finns det i intervallet $[1234, 3456]$.

\Subtask En tupel med $n$ värden kallas $n$-tupel. Om man betraktar enhetsvärdet \code{()} som en tupel, vad kan man då kalla detta värde?

\Task \emph{Objekt med attribut (fält).} Ett objekt kan samla data som hör ihop och på så sätt skapa en datastruktur. Data i ett objekt kallas \emph{attribut} eller \emph{fält}, \Eng{field}. Objekt som samlar enbart data kallas även \emph{post} \Eng{record}.
\begin{REPLnonum}
scala> object mittKonto { var saldo = 0; val nummer = 12345L }
\end{REPLnonum}
\Subtask Skriv en sats som sätter in ett slumpmässigt belopp mellan 0 och en miljon på \code{mittKonto} ovan med hjälp av punktnotation och tilldelning. 

\Subtask Vad händer om du försöker ändra attributet \code{nummer}?

\Task \emph{Klass med attribut.} Om du vill ha många objekt av samma typ, kan du använda en \textbf{klass}. På så sätt kan man skapa många datastrukturer av samma typ men med olika innehåll. Man skapar nya objekt med nyckelordet \code{new} följt av klassens namn. Klassen utgör en ''mall'' för objektet som skapas. Ett objekt som skapas med \code{new Klassnamn} kallas även en \textbf{instans} av klassen \code{Klassnamn}. Nedan skapas en datastruktur \code{Konto} som samlar data om ett bankonto. Poster av typen \code{Konto} håller reda på hur mycket pengar det finns på kontot och vilket kontonumret är:

\begin{REPL}
scala> class Konto {
         var saldo = 0
         var nummer = 0L
       }
scala> val k1 = new Konto
scala> val k2 = new Konto
scala> k1.saldo = 1000
scala> k1.nummer = 12345L
scala> k2.saldo = 2000
scala> k2.nummer = 67890L
scala> println("Konto: " + k1.nummer + " Saldo:" k1.saldo)
scala> println("Konto: " + k2.nummer + " Saldo:" k2.saldo)
\end{REPL}

\Subtask\Pen Rita hur minnessituationen ser ut efter att ovan rader har exekverats.

\Subtask\Pen Vad hade det fått för konsekvenser om attributet \code{nummer} vore oföränderligt i klassen ovan? (Jämför med objektet \code{mittKonto}.)


\Task \emph{Klass med attribut som parametrar.} Om man vill ge attributen initialvärden när objektet skaps med \code{new} kan placera attributen i en parameterlista till klassen. Koden som körs när objektet skapas och attributen tilldelas sina initaialvärden, kallas \textbf{konstruktor} \Eng{constructor}.

\begin{REPL}
scala> class Konto(var saldo: Int, val nummer: Long)
scala> val k = new Konto(0, 12345L)
scala> println("Konto: " + k.nummer + " Saldo:" k.saldo)
scala> println(k)
scala> k.toString
\end{REPL}

\Subtask Den två sista raderna ovan skriver ut den identifierare som JVM använder för att hålla reda på objektet i sina interna datastrukturer. Vad skrivs ut?

\Subtask Skapa ännu en instans av klassen Konto  med samma saldo och nummer som \code{k} ovan och spara den i \code{val k2} och undersök dess objektidentifierare. Får objekten \code{k} och \code{k2} olika objektidentifierare?

\Subtask Sätt in olika belopp på respektive konto.

\Subtask Vad händer om du försöker ändra attributet \code{nummer}?

\Subtask\Pen Ibland räcker det fint med en tupel, men ofta vill man ha en klass istället. Beskriv några fördelar med en Konto-klassen ovan jämfört med en tupel av typen \code{(Int, Long)}.

\begin{REPLnonum}
scala> var k3 = (0, 12345L)
scala> k3 = (k3._1 + 100, k3._2)
\end{REPLnonum}

\Task \emph{Publikt versus privat attribut.} Man kan förhindra att ett attribut syns utanför klassen med hjälp av nyckelordet \code{private}.  

\begin{REPL}
scala> class Konto1(val nummer: Long){ var saldo = 0 }
scala> val k1 = new Konto1(12345678901L)
scala> k1.nummer
scala> k1.saldo += 1000
scala> class Konto2(val nummer: Long){ private var saldo = 0 }
scala> val k2 = new Konto2(12345678901L)
scala> k2.nummer
scala> k2.saldo += 1000
\end{REPL}

\Subtask Vad händer ovan?

\Subtask Gör en ny version av klassen \code{Konto} enligt nedan:

\begin{Code}
class Konto(val nummer: Long){ 
  private var saldo = 0
  def in(belopp: Int): Unit = {saldo += belopp}
  def ut(belopp: Int): Unit = {saldo -= belopp}
  def show: Unit = 
    println("Konto Nr: " + nummer + " saldo: " + saldo) 
}

object Main {
  def main(args: Array[String]): Unit = {
    val k = new Konto(1234L)
    k.show
    k.in(1000)
    println("Uttag: " + k.ut(500))
    println("Uttag: " + k.ut(1000))
    k.show
  }
}
\end{Code}

\Subtask Spara koden i en fil, kompilera och kör. Testa även vad som händer om du försöker komma åt attributet \code{saldo} i main-metoden med t.ex. \code{println(k.saldo)} eller \code{k.saldo += 1000}. 

\Subtask Vi ska nu förhindra överuttag. Ändra i metoden \code{ut} så att den får signaturen \code{ut(belopp: Int): (Int Int) = ???} och implementera \code{ut} så att den returnerar både beloppet man verkligen kan ta ut och kvarvarande saldo. Om man försöker ta ut mer än det finns på kontot så ska saldot bli 0 och man får bara ut det som finns kvar. Spara, kompilera, kör. 

\Subtask Förbättra metoderna \code{in} och \code{ut} så att man inte kan sätta in eller ta ut negativa belopp.

\Subtask Vad är fördelen med att göra föränderliga attribut privata och bara påverka deras värden indirekt via metoder?

\Task \emph{Vilken typ har ett objekt?} Objektets typ bestäms av klassen. Vid tilldelning måste typerna passa ihop.

\Subtask Vilka rader nedan ger felmeddelande? Hur lyder felmeddelandet?
\begin{REPL}
scala> class Punkt(val x: Double, val y: Double)
scala> val pt: Punkt = new Punkt(10.0, 10.0)
scala> val i: Int = pt.x
scala> val (x: Double, y: Double) = (pt.x, pt.y)
scala> val p: Double = new Punkt(5.0, 5.0)
scala> val p = new Punkt(5.0, 5.0): Double
scala> val p = new Punkt(5.0, 5.0): Punkt
scala> pt: Punkt
\end{REPL}


\Subtask Man kan undersöka om ett objekt är av en viss typ med metoden \\ \code{isInstanceOf[Typnamn]}. Vad ger nedan anrop av metoden \code{isInstanceOf} för värde?
\begin{REPL}
scala> class Punkt(val x: Double, val y: Double)
scala> val pt: Punkt = new Punkt(1.0, 2.0)
scala> pt.isInstanceOf[Punkt]
scala> pt.isInstanceOf[Double]
scala> pt.x.isInstanceOf[Double]
scala> pt.x.isInstanceOf[Int]
scala> pt.x.isInstanceOf[Punkt]
\end{REPL}

\Task \emph{\code{Any}}. Alla klasser är också av typen \code{Any}. Alla klasser får därmed med sig några gemensamma metoder som finns i den fördefinierade klassen \code{Any}, däribland metoderna  \code{isInstanceOf} och \code{toString}.  Vad blir resultatet av respektive rad nedan? Vilken rad ger ett felmeddelande?

\begin{REPL}
scala> class Punkt(val x: Double, val y: Double)
scala> val pt: Punkt = new Punkt(1.0, 2.0)
scala> pt.isInstanceOf[Punkt]
scala> pt.isInstanceOf[Any]
scala> pt.x.toString
scala> println(pt.x)
scala> val a: Any = pt
scala> println(a.x)
scala> a.toString
scala> p1.y.toString
scala> a.y.toString
\end{REPL}

\Task \emph{Byta ut metoden \code{toString}}. I klassen \code{Any} finns metoden \code{toString} som skapar en strängrepresentation av objektet. Du kan byta ut metoden \code{toString} i klassen \code{Any} mot din egen implementation. Man använder nyckelordet \code{override} när man vill byta ut en metodimplementation.

\begin{REPL}
scala> class Punkt(val x: Double, val y: Double) {
         override def toString: String = "[x=" + x + ",y=" + y + "]"
       }
scala> val pt = new Punkt(1.0, 42.0)
scala> pt.toString
scala> println(pt)
\end{REPL}

\Subtask Vad händer egentligen på sista raden ovan?

\Subtask Omdefiniera toString så att den ger en sträng på formen \code{Punkt(1.0, 42.0)}.

\Subtask Vad hände om du utelämnar nyckelordet \code{override} vid omdefiniering?

\Task \emph{Objektfabrik med \code{apply}-metod.} Man kan ordna så att man slipper skriva \code{new} med ett s.k. \emph{fabriksobjekt} \Eng{factory object}. 
\begin{Code}
class Pt(val x: Double, y: Double) {
  override def toString: String = "Pt(x=" + x + ",y=" + y + ")"
}
object Pt { 
  def apply(x: Double, y: Double): Pt = new Pt(x, y)
}
\end{Code}

\Subtask Skriv satser som använder metoden \code{apply} i fabriksobjektet \code{object Pt} för att skapa flera olika punkter.

\Subtask Ge applymetoden default-argument 0.0 för både x och y så att \code{Pt()} skapar en punkt i origo.

\Subtask Skapa en klass \code{Rational} som representerar rationellt tal som en kvot mellan två heltal. Ge klassen två oföränderliga, publika klassparameterattribut med namnen \code{nom} för täljaren och \code{denom} för nämnaren. 

\Subtask Skapa ett fabriksobjekt med en \code{apply}-metod som tar två heltalsparametrar och skapar en instans av klassen \code{Rational}.

\Subtask Skapa olika instanser av din klass \code{Rational} ovan med hjälp av fabriksobjektet.


\Task \emph{Skapa en case-klass.} Med en case-klass får man \code{toString} och fabriksobjekt på köpet. Man behöver inte skriva \code{val} framför klassparametrar i case-klasser; klassparametrar blir publika, oföränderliga attribut automatiskt när man deklarerar en case-klass.

\begin{REPL}
scala> case class Pt(x: Double, y: Double) 
scala> val p = Pt(1.0, 42.0)
scala> p.toString
scala> println(p)
scala> println(Pt(5,6))
\end{REPL}

\Subtask Implementera din klass \code{Rational} från föregående uppgift, men nu som en case-klass.

\Subtask Skapa en case-klass \code{Complex} som representerar komplext tal. Ge klassen två oföränderliga, publika Double-attribut: \code{re} som lagrar realdelen och \code{im} som lagrar imaginärdelen. 

\Task \label{task:point} \emph{Metoder på datastrukturer.} En datastruktur blir mer användbar om det finns metoder som kan användas på datastrukturen. Metoder i Scala kan även ha (vissa) tecken som namn, t.ex. \code{+} enligt nedan.  
\begin{REPL}
scala> case class Point(x: Double, y: Double) {
         def length: Double = math.hypot(x, y)   
         def add(p: Point): Point = Point(x + p.x, y + p.y)
         def +(p: Point): Point = Point(x + p.x, y + p.y)
       }
\end{REPL}

\Subtask Använd metoden \code{lenght} för att ta reda på vad punkten med koordinaterna (3, 4) har för avstånd till origo?

\Subtask Skriv satser som skapar två punkter (3,4) och (5, 6) och låt variablerna p1 och p2 referera till respektive punkt. Låt variabeln p3 bli summan av p1 och p2. Vad får uttrycken p3.x resp. p3.y för värden?

\Task \emph{Operatornotation.} Vid punktnotation på formen: \\ \code{objekt.metod(parameter)} \\ där man anropar en metod med exakt en parameter, kan man skippa punkten och parenteserna och skriva:\\ \code{objekt metod parameter}  \\
Detta förenklade skrivsätt kallas \textbf{operatornotation}.

\Subtask Använd klassen \code{Point} från uppgift \ref{task:point} och prova nedan satser. Vilka rader använder operatortnotation och vilka rader använder punktnotation?
\begin{REPL}
scala> val p1 = Point(3,4)
scala> val p2 = Point(3,4)
scala> p1.add(p2)
scala> p1 add p2
scala> p1.+(p2)
scala> p1 + p2
scala> 42 + 1
scala> 42.+(1)
scala> 1.to(42)
scala> 1 to 42
\end{REPL}

\Subtask Implementera metoderna \code{sub} och \code{-} i klassen \code{Point} och skriv uttryck som kombinerar add och sub, samt + och - i både punktnotation och operatornotation.



\Task \emph{Föränderlighet och oföränderlighet.} Oföränderliga och föränderliga objekt beter sig olika vid tilldelning.  

\Subtask\Pen Innan du kör nedan kod: Försök lista ut vad som kommer att skrivas ut. Rita minnessituationen efter varje tilldelning. 

\begin{Code}
println("\n--- Example 1: mutable value assigmnent")
var x1 = 42
var y1 = x1
x1 = x1 + 42
println(x1)
println(y1)
\end{Code}

\Subtask\Pen Innan du kör nedan kod: Försök lista ut vad som kommer att skrivas ut. Rita minnessituationen efter varje tilldelning.

\begin{Code}
println("\n--- Example 2: mutable object reference assignment")
class MutableInt(private var i: Int) {
  def +(a: Int): MutableInt = { i = i + a; this }
  override def toString: String = i.toString
}
var x2 = new MutableInt(42)
var y2 = x2
x2 = x2 + 42
println(x2)
println(y2)
\end{Code}

\Subtask\Pen Innan du kör nedan kod: Försök lista ut vad som kommer att skrivas ut. Rita minnessituationen efter varje tilldelning.

\begin{Code}
println("\n--- Example 3: immutable object reference assignment")
class ImmutableInt(val i: Int) {
  def +(a: Int): ImmutableInt = new ImmutableInt(i + a) 
  override def toString: String = i.toString
}
var x3 = new ImmutableInt(42)
var y3 = x3
x3 = x3 + 42
println(x3)
println(y3)
\end{Code}

\Subtask\Pen Vad finns det för fördelar med oföränderliga datastrukturer?


\Task \emph{Några anvädnbara samlingar.} En \textbf{samling} \Eng{collection} är en datastruktur som samlar många objekt av samma typ. I \code{scala.collection} och \code{java.util} finns många olika samlingar med en uppsjö användbara metoder. De olika samlingarna i \code{scala.collection} är ordnade i en gemensam hierarki med många gemensamma metoder; därför har man nytta av det man lär sig om metoderna i en Scala-samling när man använder en annan samling. Vi har redan tidigare sett samlingen \code{Vector}:

\begin{REPL}
scala> val tärningskast = Vector.fill(10000)((math.random * 6 + 1).toInt)
scala> tä   // tryck TAB
scala> tärningskast.  // tryck TAB
\end{REPL}

\Subtask Ungefär hur många metoder finns det som man kan göra på objekt av typen \code{Vector}? Det är svårt att lära sig alla dessa på en gång, så vi väljer ut några få i kommande uppgifter.

\Subtask Jämför överlappet mellan metoderna i \code{Vector} och \code{List} och uppskatta hur stor andel av metoderna som är gemensamma: 
\begin{REPL}
scala> val myntkast = 
         List.fill(10000)(if (math.random < 0.5) "krona" else "klave")
scala> my   // tryck TAB
scala> myntkast.  // tryck TAB
\end{REPL}

\Task \emph{Några viktiga samlingsmetoder.} Deklarera följande vektorer i REPL. 
\begin{REPL}
scala> val xs = (1 to 10).toVector
scala> val a = Vector("abra", "ka", "dabra")
scala> val b = Vector( "sim", "sala", "bim", "sala", "bim")
scala> val stor = Vector.fill(100000)(math.random)
\end{REPL}
Undersök i REPL vad som händer nedan. Alla dessa metoder fungerar på alla samlingar som är indexerbara sekvenser. Givet deklarationerna ovan: vad har uttrycken nedan för värde och typ? Förklara vad som händer hälp av denna  översikt: \href{http://docs.scala-lang.org/overviews/collections/seqs}{docs.scala-lang.org/overviews/collections/seqs}

\Subtask \code{a(1) + xs(1)}

\Subtask \code{a apply 0}

\Subtask \code{a.isDefinedAt(3)}

\Subtask \code{a.isDefinedAt(100)}

\Subtask \code{stor.length}

\Subtask \code{stor.size}

\Subtask \code{a indexOf "ka"}

\Subtask \code{b.lastIndexOf("sala")}

\Subtask \code{"först" +: b}

\Subtask \code{a :+ "sist"}

\Subtask \code{xs.updated(2,42)}

\Subtask \code{a.padTo(10, "!")}

\Subtask \code{b.sorted}

\Subtask \code{b.reverse}

\Subtask \code{a.startsWith(Vector("abra", "ka"))}

\Subtask \code{"hejsan".endsWith("san")}

\Subtask \code{b.distinct}




\Task \emph{Några genrella samlingsmetoder.} Det finns metoder som går att köra på \emph{alla} samlingar även om de inte är indexerbara. Givet deklarationerna i föregående uppgift: vad har uttrycken nedan för värde och typ? Förklara vad som händer med hjälp av dessa översikter: \\ \href{http://docs.scala-lang.org/overviews/collections/trait-traversable}{docs.scala-lang.org/overviews/collections/trait-traversable} \\ \href{http://docs.scala-lang.org/overviews/collections/trait-iterable}{docs.scala-lang.org/overviews/collections/trait-iterable}

\Subtask \code{a ++ b}

\Subtask \code{a ++ stor}

\Subtask \code{val ys = xs.map(_ * 5)}

\Subtask \code{}

\Subtask \code{}

\Subtask \code{}

\Subtask \code{}

\Subtask \code{}	

\Subtask \code{}

\Subtask \code{}

\Subtask \code{}

\Subtask \code{}

\Subtask \code{}	





\Task De olika samlingarna i \code{scala.collection} används flitigt i andra paket, exempelvis \code{scala.util} och \code{scala.io}. 

\Subtask Vad händer här? (Metoden \code{shuffle} skapar en ny samling med elementen i slumpvis ordning.)
\begin{Code}
val xs = Vector(1,2,3)
def blandat = scala.util.Random.shuffle(xs)
def test = if (xs == blandat) "lika" else "olika"
(for(i <- 1 to 100) yield test).count(_ == "lika")
\end{Code}


\Subtask Skapa en textfil med namnet \code{fil.txt} som innehåller lite text och läs in den med: \\\code{scala.io.Source.fromFile("fil.txt", "UTF-8").getLines.toVector}
\begin{REPL}
> cat > fil.txt
hejsan
svejsan
> scala
scala> val xs = scala.io.Source.fromFile("fil.txt", "UTF-8").getLines.toVector
scala> xs.foreach(println)
\end{REPL}


\Subtask Vad händer här? (Metoden \code{trim} på värden av typen \code{String} ger en ny sträng med blanktecken i början och slutet borttagna.) 
\begin{REPL}
scala> val pgk = scala.io.Source.fromURL("http://cs.lth.se/pgk/","UTF-8").getLines.toVector
scala> pgk.map(_.trim).filterNot(_.startsWith("<")).filterNot(_.isEmpty).foreach(println)
\end{REPL}



\Task \emph{Jämföra List och Vector.} En indexerbar sekvens av värden kallas vektor eller lista. I Scala finns flera klasser som kan kan indexeras, däribland klasserna \code{Vector} och \code{List}. 

\Subtask \emph{Likheter mellan \code{Vector} och \code{List}.} Kör nedan rader i REPL. Prova indexera i båda och studera hur stor andel av metoderna som är gemensamma.
\begin{REPL}
scala> val sv = Vector("en", "två", "tre", "fyra")
scala> val en = List("one", "two", "three", "four")
scala> sv(0) + sv(3)
scala> en(0) + en(3)
scala> sv. //tryck TAB 
scala> en. //tryck TAB
\end{REPL}

\Subtask \emph{Skillnader mellan \code{Vector} och \code{List}.} Klassen \code{Vector} i Scala har ''under huven'' en avancerad datastruktur i form av ett s.k. självbalanserande träd, vilket gör att \code{Vector} är snabbare än \code{List} på nästan allt, \emph{utom} att bearbeta elementen i \emph{början} av sekvensen; vill man lägga till och ta bort i början av en \code{List} så kan det ibland gå ungefär dubbelt så fort jämfört med \code{Vector}, medan alla andra operationer är lika snabba eller snabbare med \code{Vector}. Det finns ett fåtal speciella metoder, som bara finns i \code{List}, för att skapa en lista och lägga till i början av en lista. 

\begin{REPL}
scala> var xs = "one" :: "two" :: "three" :: "four" :: Nil
scala> xs = "zero" :: xs
scala> val ys = xs.reverse ::: xs
\end{REPL}


\Task \emph{Mängd.} En mängd är en samling som garanterar att det inte finns några dubbletter. Det går dessutom snabbt att kolla om ett element finns eller inte i en mängd. Elementen i samlingen \code{Set} hamnar ibland, av effektivitetsskäl, i en förvånande ordning.
\begin{REPL}
scala> val s = Set("Malmö", "Stockolm", "Göteborg", "Köpenhamn", "Oslo")
s: scala.collection.immutable.Set[String] = 
     Set(Oslo, Malmö, Köpenhamn, Stockolm, Göteborg)

scala> val t = Set("Sverige", "Sverige", "Sverige", "Danmark", "Norge")
t: scala.collection.immutable.Set[String] = Set(Sverige, Danmark, Norge)
\end{REPL}
Givet ovan deklarationer: vad blir värde och typ av nedan uttryck?

\Subtask \code{s + "Malmö" == s}

\Subtask \code{s ++ t}

\Subtask \code{Set("Malmö", "Oslo").subsetOf(s)}

\Subtask \code{s subsetOf Set("Malmö", "Oslo")}

\Subtask \code{s contains "Lund"}

\Subtask \code{s apply "Lund"}

\Subtask \code{s("Malmö")}

\Subtask \code{s - "Stockholm"}

\Subtask \code{t - ("Norge", "Danmark", "Tyskland")}

\Subtask \code{s -- t}

\Subtask \code{s -- Set("Malmö", "Oslo")}

\Subtask \code{Set(1,2,3) intersect Set(2,3,4)}

\Subtask \code{Set(1,2,3) & Set(2,3,4)}

\Subtask \code{Set(1,2,3) union Set(2,3,4)}

\Subtask \code{Set(1,2,3) | Set(2,3,4)}


\Task \emph{Slå upp värden från nycklar med Map.} Samlingen \code{Map} är mycket användbar. Med den kan man snabbt leta upp ett värde om man har en nyckel. Samlingen \code{Map} är en generalisering av en vektor, där man kan ''indexera'', inte bara med ett heltal, utan med vilken typ av värde som helst, t.ex. en sträng.
\begin{REPL}
scala> var huvudstad = 
  Map("Sverige" -> "Stockholm", "Norge" -> "Oslo", "Skåne" -> "Malmö") 
\end{REPL}
Givet ovan variabel \code{huvudstad}, vad händer nedan?

\Subtask \code{huvudstad apply "Skåne"}

\Subtask \code{huvudstad("Skåne")}

\Subtask \code{huvudstad.contains("Skåne")}

\Subtask \code{huvudstad.contains("Malmö")}

\Subtask \code{huvudstad += "Danmark" -> "Köpenhamn"}

\ExtraTasks %%%%%%%%%%%%%%%%%%%


\Task Träna mer på  klass

\begin{Code}
class Account(val number: Long, val maxCredit: Int){ 
  private var balance = 0
  
  def deposit(amount: Int): Int = { 
    if (amount > 0) {balance += amount}
    balance
  }
  
  def withdraw(amount: Int): (Int, Int) = if (amount > 0) { 
    val allowedWithdrawal = 
      if (amount < balance + maxCredit) amount 
      else balance + maxCredit 
    balance = balance - allowedWithdrawal
    (allowedWithdrawal, balance)
  } else (0, balance)
  
  def show: Unit = 
    println("Account Nbr: " + number + " balance: " + balance) 
}

object Main {
  def main(args: Array[String]): Unit = {
    ???
  }
}
\end{Code}



\Task Träna mer på mängd  

\Subtask Keno-bollar.






\AdvancedTasks %%%%%%%%%%%%%%%%%

\Task \emph{Dokumentationen för \code{Any}.} Undersök vilka metoder som finns i klassen Any här: \href{http://www.scala-lang.org/api/current/\#scala.Any}{http://www.scala-lang.org/api/current/\#scala.Any}. Prova några av metoderna i REPL.

\Task \emph{Dokumentationen för samlingar.} Leta upp metoden \code{tabulate} i dokumentationen för objektet \code{Vector} nästan längst ner i listan här: \\ \href{http://www.scala-lang.org/api/current/#scala.collection.immutable.Vector$}{http://www.scala-lang.org/api/current/\#scala.collection.immutable.Vector\$} \\Leta upp den variant av \code{tabulate} som har signaturen:\\ \code{def tabulate[A](n: Int)(f: (Int) => A): Vector[A] }\\ Klicka på den gråfyllda trekanten till vänster om signaturen som fäller ut beskrivningen

\Subtask Förklara vad som händer här:
\begin{REPLnonum}
scala> Vector.tabulate(10)(i => i % 3)
\end{REPLnonum}

\Subtask Klicka på det blåa stora o-et överst på sidan, för att växla till klass-vyn och studera listan med alla metoder  i klassen \code{Vector}. 


\Task \emph{Fler metoder på indexerbara sekvenser.} Deklarera följande vektorer i REPL. 
\begin{REPL}
scala> val xs = (1 to 10).toVector
scala> val a = Vector("abra", "ka", "dabra")
scala> val b = Vector( "sim", "sala", "bim", "sala", "bim")
\end{REPL}
Undersök i REPL vad som händer nedan. Alla dessa metoder fungerar på alla samlingar som är indexerbara sekvenser. Vad har uttrycken för värde och typ? Förklara vad metoden gör. Studera även denna  översikt: \href{http://docs.scala-lang.org/overviews/collections/seqs}{docs.scala-lang.org/overviews/collections/seqs}

\Subtask \code{b.indexWhere(s => s.startsWith("b"))}  % advanced

\Subtask \code{a.indices}  % advanced

\Subtask \code{xs.patch(1, Vector(42,43,44), 7)} % advanced

\Subtask \code{xs.segmentLength(_ < 8, 2)} % advanced

\Subtask \code{b.sortBy(_.reverse)}  % advanced

\Subtask \code{b.sortWith((s1, s2) => s1.size < s2.size)} % advanced

\Subtask \code{a.reverseMap(_.size)}	% advanced

\Subtask \code{a intersect Vector("ka", "boom", "pow")} % advanced

\Subtask \code{a diff Vector("ka")} % advanced

\Subtask \code{a union Vector("ka", "boom", "pow")} % advanced



\Task Jämför tidsprestanda mellan List och Vector vid hantering i början och i slutet. 

\Subtask Hur snabbt går nedan på din dator? (Exemplet nedan är exekverat på en Intel i7-4790K CPU @ 4.00GHz.) %sudo lshw -class processor

\begin{REPLnonum}
scala> :paste

def time(n: Int)(block: => Unit): Double =  {
  def now = System.nanoTime
  var timestamp = now
  var sum = 0L
  var i = 0
  while (i < n) {
    block
    sum = sum + (now - timestamp)
    timestamp = now
    i = i + 1
  }
  val average = sum.toDouble / n
  println("Average time: " + average + " ns")
  average
}

scala> val n = 100000
scala> val l = List.fill(n)(math.random)
scala> val v = Vector.fill(n)(math.random)

scala> (for(i <- 1 to 20) yield time(n){l.take(10)}).min
Average time: 47.1852 ns
Average time: 41.64156 ns
Average time: 105.53986 ns
Average time: 41.91562 ns
Average time: 41.73559 ns
Average time: 63.17134 ns
Average time: 52.93756 ns
Average time: 41.58533 ns
Average time: 41.68017 ns
Average time: 60.18881 ns
Average time: 41.69867 ns
Average time: 41.60771 ns
Average time: 60.32759 ns
Average time: 41.62671 ns
Average time: 43.88916 ns
Average time: 70.47824 ns
Average time: 41.68801 ns
Average time: 41.67223 ns
Average time: 41.67262 ns
Average time: 102.84893 ns
res85: Double = 41.58533

scala> (for(i <- 1 to 20) yield time(n){v.take(10)}).min
Average time: 312.67005 ns
Average time: 88.60023 ns
Average time: 73.21829 ns
Average time: 92.148 ns
Average time: 91.01078 ns
Average time: 87.82874 ns
Average time: 74.04663 ns
Average time: 94.16038 ns
Average time: 88.4243 ns
Average time: 105.88971 ns
Average time: 98.85731 ns
Average time: 72.77369 ns
Average time: 97.04337 ns
Average time: 90.01969 ns
Average time: 88.11196 ns
Average time: 75.20191 ns
Average time: 93.72112 ns
Average time: 110.19777 ns
Average time: 132.4207 ns
Average time: 324.28702 ns
res86: Double = 72.77369

scala> (for(i <- 1 to 20) yield time(1000){l.takeRight(10)}).min
Average time: 247365.43 ns
Average time: 212801.958 ns
Average time: 212335.938 ns
Average time: 212313.427 ns
Average time: 212524.963 ns
Average time: 219525.627 ns
Average time: 223059.563 ns
Average time: 222426.504 ns
Average time: 221838.828 ns
Average time: 223268.567 ns
Average time: 222739.402 ns
Average time: 222685.229 ns
Average time: 223122.599 ns
Average time: 222683.921 ns
Average time: 222865.865 ns
Average time: 222889.118 ns
Average time: 223247.135 ns
Average time: 222016.82 ns
Average time: 223040.299 ns
Average time: 222624.613 ns
res87: Double = 212313.427

scala> (for(i <- 1 to 20) yield time(1000){v.takeRight(10)}).min
Average time: 2665.715 ns
Average time: 190634.043 ns
Average time: 773.111 ns
Average time: 509.008 ns
Average time: 519.04 ns
Average time: 418.172 ns
Average time: 365.54 ns
Average time: 409.016 ns
Average time: 353.115 ns
Average time: 503.679 ns
Average time: 421.369 ns
Average time: 388.685 ns
Average time: 461.725 ns
Average time: 390.791 ns
Average time: 381.83 ns
Average time: 309.667 ns
Average time: 372.09 ns
Average time: 312.254 ns
Average time: 323.925 ns
Average time: 310.261 ns
res88: Double = 309.667

\end{REPLnonum}

\Subtask Varför går det olika snabbt olika körningar?

\Task Studera skillnader i prestanda mellan olika samlingar här: \\ \href{http://docs.scala-lang.org/overviews/collections/performance-characteristics.html}{docs.scala-lang.org/overviews/collections/performance-characteristics.html} \\
(Mer om detta i kommande kurser.)

\Task Gör något rekursivt med en lista för att visa hur syntaxen kan se ut med cons.
    
%!TEX encoding = UTF-8 Unicode

%!TEX root = ../compendium.tex

\Lab{\LabWeekFOUR}

\begin{Goals}
\item Att lära sig.
\end{Goals}

\begin{Preparations}
\item Att göra.
\end{Preparations}

\subsection{Obligatoriska uppgifter}

\Task En labbuppgiftsbeskrivning.

\Subtask En underuppgift.

\Subtask En underuppgift.

\subsection{Frivilliga extrauppgifter}

\Task En labbuppgiftsbeskrivning.

\Subtask En underuppgift.

\Subtask En underuppgift.
    
%!TEX encoding = UTF-8 Unicode

%!TEX root = ../compendium.tex

\input{generated/w05-chaphead-generated.tex}
\clearpage

\input{../slides/body/lect-week05-seqalg.tex}
    
%!TEX encoding = UTF-8 Unicode

%!TEX root = ../compendium.tex

\Exercise{\ExeWeekFIVE}

\begin{Goals}
\item 
\end{Goals}

\begin{Preparations}
\item 
\end{Preparations}

\BasicTasks %%%%%%%%%%%%%%%%

\Task \emph{Variabelt antal argument.} Det går fint att deklarera en funktion som tar en argumentsekvens av godtycklig längd. Syntaxen består ev en asterisk \code{*} efter typen.

\Subtask Vad händer nedan?
\begin{REPL}
scala> def printAll(xs: Int*) = xs.foreach(println)
scala> printAll(42)
scala> printAll(1, 2, 7, 42)
scala> def printStrings(wa: String*) = println(wa)
scala> printStrings("hej","på","dej")
\end{REPL}

\Subtask Vad har parametern \code{wa} i \code{printStrings} ovan för typ?

\Subtask Ändra i \code{printAll} så att även längden på \code{xs} skrivs ut före utskriften av alla element. Testa att anropa \code{printAll} med olika antal parametrar. 

\Subtask Vad händer om du anropar \code{printAll} med noll parametrar?

\Task \emph{Oföränderliga sekvenser med föränderliga objekt.} 

\Subtask Vad får xs för värde efter att attributet i objektet som \code{c2} refererar till ändras på rad 4 nedan? Förklara vad som händer.
\begin{REPL}
scala> class IntCell(var x: Int){override def toString = "[Int](" + x + ")"}
scala> val (c1, c2, c3) = (new IntCell(7), new IntCell(8), new IntCell(9))
scala> val xs = Vector(c1, c2, c3)
scala> c2.x = 42
scala> xs
\end{REPL}

\Subtask\Pen Rita en bild av minnessituationen efter rad 4 ovan.

\Subtask\Pen Vad krävs för att allt innehåll i en oföränderlig samling garanterat ska förbli oförändrat? 

\Task Föränderliga, indexerbara sekvenser: \code{Array} och \code{ArrayBuffer}

\Subtask Samlingen \code{scala.Array} har speciellt stöd i JVM och är extra snabb att allokera och indexera i. Dock kan man inte ändra storleken efter att en Array allokerats. Behöver man mer plats kan man kopiera den till en ny, större array. Koden nedan visar hur det kan gå till.
\begin{REPL}
scala> val xs = Array(42, 43, 44)
scala> val ys = new Array[Int](4)  //plats för 4 heltal, från början nollor
scala> for (i <- 0 until xs.size){ys(i) = xs(i)}
scala> ys(3) = 45
\end{REPL}
Definiera funktionen \code{def copyAppend(xs: Array[Int], x): Array[Int]} som implementerar nedan algoritm, \emph{efter} att du rätta de \textbf{\color{red}{två buggarna}} i algoritmens while-loop:

\begin{algorithm}[H]
 \SetKwInOut{Input}{Indata}\SetKwInOut{Output}{Resultat}
 
 \Input{Heltalsarray $xs$ och heltalet $x$}
 \Output{En ny array som som är en kopia av $xs$ men med $x$ tillagt på slutet som extra element.}
 $n \leftarrow$ antalet element i $xs$ \\
 $ys \leftarrow$ en ny array med plats för $n + 1$ element\\
 $i \leftarrow 0$  \\
 \While{$i \leq n$}{
  $ys(i) \leftarrow xs(i)$
 }
 $ys(n) \leftarrow x$ 
\end{algorithm}



\Subtask Samlingen \code{scala.collection.mutable.ArrayBuffer} är inte riktigt lika snabb i alla lägen som \code{scala.Array} men storleksändring hanteras automatiskt, vilket är en stor fördel då man slipper att själv implementera algoritmer liknande \code{copyAppend} ovan. Speciellt använder man ofta \code{ArrayBuffer} om man stegvis vill bygga upp en sekvens. Vad händer nedan?
\begin{REPL}
scala> val xs = scala.collection.mutable.ArrayBuffer.empty[Int]
scala> xs.append(1, 2)
scala> while (xs.last < 100) {xs.append(xs.takeRight(2).sum); println(xs)}
scala> xs.last
scala> xs.length
\end{REPL}

\Subtask Talen i sekvensen som produceras ovan kallas Fibonaccital\footnote{\href{https://sv.wikipedia.org/wiki/Fibonaccital}{sv.wikipedia.org/wiki/Fibonaccital}}. Hur lång ska en Fibonacci-sekvens vara för att det sista elementet ska komma så nära (men inte över) \code{Int.MaxValue} som möjligt?



\Task \emph{Kopiering och uppdatering.} Metoder på oföränderliga samlingar skapar nya samlingar istället för att ändra. Därför behöver man inte själv skapa kopior. När en \emph{föränderlig} samling uppdateras på plats, syns denna förändring via alla referenser till samlingen.

\begin{REPL}
scala> val xs = Vector(1, 2, 3)
scala> val ys = xs.toArray
scala> ys(1) = 42
scala> xs
scala> ys
scala> val zs = ys.toArray
scala> zs(1) = 84
scala> xs
scala> ys
scala> zs
\end{REPL}

\Subtask Syns updateringen av objektet som \code{ys} refererar till via referensen \code{xs}? Varför?

\Subtask Syns updateringen av objektet som \code{zs} refererar till via referensen \code{ys}? Varför? 

\Subtask Syns updateringen av objektet som \code{zs} refererar till via referensen \code{xs}? Varför?

\Task \emph{Färdig metod för att skapa kopia av array.} Om man inte vill att en uppdatering av en föränderlig samling ska få oönskad påverkan på andra koddelar som refererar till samlingen, behöver man göra kopior av samlingen före uppdatering. Det finns färdiga metoder för kopiering av objekt av typen Array i paketet \code{java.util.Arrays}. 

\Subtask\Pen Studera dokumentationen för metoden \code{java.util.Arrays.copyOf} här:\\ \href{https://docs.oracle.com/javase/8/docs/api/java/util/Arrays.html\#copyOf-int:A-int-}{docs.oracle.com/javase/8/docs/api/java/util/Arrays.html\#copyOf-int:A-int-} \\
Notera att syntaxen för arrayer i Java är annorlunda: När det står \code{int[]} i Java så motsvarar det \code{Array[Int]} i Scala. Vad används den andra parametern till?

\Subtask\Pen Rita en bild av hur minnet ser ut efter varje tilldelning nedan. Vad har \code{xs}, \code{ys} och \code{zs} för värden efter exekveringen av raderna 1--5 nedan? Varför? 
\begin{REPL}
scala> val xs = Array(1, 2, 3, 4)
scala> val ys = xs
scala> val zs = java.util.Arrays.copyOf(xs, xs.size - 1)
sxala> xs(0) = 42
scala> zs(0) = 84
scala> ys
scala> xs
scala> zs
\end{REPL}

\Task \emph{Algortim: SEQ-REVERSE-COPY.} Implementera nedan algoritm:

\begin{algorithm}[H]
 \SetKwInOut{Input}{Indata}\SetKwInOut{Output}{Resultat}
 
 \Input{Heltalsarray $xs$ och heltalet $x$}
 \Output{En ny heltalsarray med elementen i $xs$ i omvänd ordning.}
 $n \leftarrow$ antalet element i $xs$ \\
 $ys \leftarrow$ en ny heltalsarray med plats för $n$ element\\
 $i \leftarrow 0$  \\
 \While{$i < n$}{
  $ys(n - i - 1) \leftarrow xs(i)$ \\
  $i \leftarrow i + 1$
 }
 \Return $ys$
\end{algorithm}

\Subtask\Pen Skriv implementation med penna och papper. Använd en \code{while}-sats på samma sätt som i algoritmen. Prova sedan din implementation på dator och kolla så att den fungerar.

\Subtask\Pen \label{subtask:for-seq-copy} Skriv implementationen med penna och papper igen, men använd nu istället en \code{for}-sats som räknar baklänges. Prova sedan din implementation på dator och kolla så att den fungerar. 

\Subtask Definiera en funktion i REPL med namnet \code{reverseCopy} med din implementation i uppgift \ref{subtask:for-seq-copy}.  


\Task \emph{Algortim: SEQ-REVERSE.} Strängar av typen \code{String} är oföränderliga. Vill man ändra i en sträng utan att skapa en ny kopia kan man använda en \code{StringBuffer} enligt nedan algoritm som vänder bak-och-fram på en sträng. 

\begin{algorithm}[H]
 \SetKwInOut{Input}{Indata}\SetKwInOut{Output}{Resultat}
 
 \Input{En sträng $s$ av typen \texttt{String}}
 \Output{En ny sträng av typen \texttt{String}}
 $sb \leftarrow$ en ny \texttt{StringBuilder} som innehåller $s$ \\
 $n \leftarrow$ antalet tecken i $s$\\
 $i \leftarrow 0$  \\
 \For{$i \leftarrow 0$ \KwTo $\frac{n}{2} - 1$}{
  $temp \leftarrow sb(i)$ \\
  $sb(i) \leftarrow sb(n - i - 1)$ \\
  $sb(n - i - 1) \leftarrow temp$ \\
 }
 \Return $sb$ omvandlad till en \texttt{String}
\end{algorithm}

\Subtask Implementera algoritmen ovan i en funktion med signaturen: \\
 \code{def reverseString(s: String): String}

\begin{Code}
// Kod till facit:
def reverseString(s: String): String = {
  val sb = new StringBuilder(s)
  val n = sb.length
  for (i <-0 until n / 2) { 
    val temp = sb(i)
    sb(i) = sb(n - i - 1)
    sb(n - i - 1) = temp
  }
  sb.toString     
}
\end{Code}

\Subtask Använd din funktion \code{reverseString} från föregående deluppgift i en ny funktion med signaturen:\\
 \code{def isPalindrome(s: String): Boolean} \\ som avgör om en sträng är en palindrom.\footnote{\href{https://sv.wikipedia.org/wiki/Palindrom}{sv.wikipedia.org/wiki/Palindrom}} 

\Subtask\Pen Man kan med en \code{while}-sats och indexering direkt i en \code{String} avgöra om en sträng är en palindrom utan att kopiera den till en \code{StringBuilder}. Implementera en ny variant av \code{isPalindrome} som använder denna metod. Skriv först algoritmen på papper i pseudo-kod.

\begin{Code}
// Kod till facit:
def isPalindrome(s: String): Boolean = {
  val n = s.length
  var foundDiff = false
  var i = 0
  while (i < n/2 && !foundDiff)  { 
    foundDiff = s(i) != s(n - i - 1)
    i += 1
  }
  !foundDiff
}
\end{Code}

\Task Keno-dragningar under ett år -> Registrering...

\ExtraTasks %%%%%%%%%%%%%%%%%%%






\AdvancedTasks %%%%%%%%%%%%%%%%%

\Task Studera skillnader och likheter mellan 

\Subtask \code{Array}

\Subtask \code{WrappedArray}  

\Subtask \code{ArraySeq} 

\noindent genom att läsa mer om dessa arrayvarianter här: \\
\href{http://docs.scala-lang.org/overviews/collections/concrete-mutable-collection-classes}{docs.scala-lang.org/overviews/collections/concrete-mutable-collection-classes} \\  
\href{http://docs.scala-lang.org/overviews/collections/arrays.html}{docs.scala-lang.org/overviews/collections/arrays.html}  \\ 
\href{http://stackoverflow.com/questions/5028551/scala-array-vs-arrayseq}{stackoverflow.com/questions/5028551/scala-array-vs-arrayseq}   
    
    
\Task Studera vad metoden \code{java.util.Arrays.deepEquals} gör här:\\
\href{https://docs.oracle.com/javase/8/docs/api/java/util/Arrays.html#deepEquals-java.lang.Object:A-java.lang.Object:A-}{Arrays.html\#deepEquals-java.lang.Object:A-java.lang.Object:A-} \\
Vad skiljer ovan metod från metoden \code{java.util.Arrays.equals}?
    
%!TEX encoding = UTF-8 Unicode

%!TEX root = ../compendium.tex

\Lab{\LabWeekFIVE}

\begin{Goals}
\item Att lära sig.
\end{Goals}

\begin{Preparations}
\item Att göra.
\end{Preparations}

\subsection{Obligatoriska uppgifter}

\Task En labbuppgiftsbeskrivning.

\Subtask En underuppgift.

\Subtask En underuppgift.

\subsection{Frivilliga extrauppgifter}

\Task En labbuppgiftsbeskrivning.

\Subtask En underuppgift.

\Subtask En underuppgift.
    
%!TEX encoding = UTF-8 Unicode

%!TEX root = ../compendium.tex

\input{generated/w06-chaphead-generated.tex}
    
%!TEX encoding = UTF-8 Unicode

%!TEX root = ../compendium.tex

\Exercise{\ExeWeekSIX}

\begin{Goals}
\item 
\end{Goals}

\begin{Preparations}
\item 
\end{Preparations}

\BasicTasks %%%%%%%%%%%%%%%%

\Task 

\Subtask 

\ExtraTasks %%%%%%%%%%%%%%%%%%%

\Task 

\AdvancedTasks %%%%%%%%%%%%%%%%%

\Task     
    
%!TEX encoding = UTF-8 Unicode

%!TEX root = ../compendium.tex

\Lab{\LabWeekSIX}

\begin{Goals}
\item Att lära sig.
\end{Goals}

\begin{Preparations}
\item Att göra.
\end{Preparations}

\subsection{Obligatoriska uppgifter}

\Task En labbuppgiftsbeskrivning.

\Subtask En underuppgift.

\Subtask En underuppgift.

\subsection{Frivilliga extrauppgifter}

\Task En labbuppgiftsbeskrivning.

\Subtask En underuppgift.

\Subtask En underuppgift.
    
%!TEX encoding = UTF-8 Unicode

%!TEX root = ../compendium.tex

\input{generated/w07-chaphead-generated.tex}
    
%!TEX encoding = UTF-8 Unicode

%!TEX root = ../compendium.tex

\Exercise{\ExeWeekSEVEN}

\begin{Goals}
\item 
\end{Goals}

\begin{Preparations}
\item 
\end{Preparations}

\BasicTasks %%%%%%%%%%%%%%%%

\Task 

\Subtask 

\ExtraTasks %%%%%%%%%%%%%%%%%%%

\Task 

\AdvancedTasks %%%%%%%%%%%%%%%%%

\Task     
    
%!TEX encoding = UTF-8 Unicode

%!TEX root = ../compendium.tex

\Lab{\LabWeekSEVEN}

\begin{Goals}
\item Att lära sig.
\end{Goals}

\begin{Preparations}
\item Att göra.
\end{Preparations}

\subsection{Obligatoriska uppgifter}

\Task En labbuppgiftsbeskrivning.

\Subtask En underuppgift.

\Subtask En underuppgift.

\subsection{Frivilliga extrauppgifter}

\Task En labbuppgiftsbeskrivning.

\Subtask En underuppgift.

\Subtask En underuppgift.
    
%!TEX encoding = UTF-8 Unicode

%!TEX root = ../compendium.tex

\input{generated/w08-chaphead-generated.tex}
    
%!TEX encoding = UTF-8 Unicode

%!TEX root = ../compendium.tex

\Exercise{\ExeWeekEIGHT}

\begin{Goals}
\item 
\end{Goals}

\begin{Preparations}
\item 
\end{Preparations}

\BasicTasks %%%%%%%%%%%%%%%%

\Task 

\Subtask 

\ExtraTasks %%%%%%%%%%%%%%%%%%%

\Task 

\AdvancedTasks %%%%%%%%%%%%%%%%%

\Task     
    
%!TEX encoding = UTF-8 Unicode

%!TEX root = ../compendium.tex

\Lab{\LabWeekEIGHT}

\begin{Goals}
\item Att lära sig.
\end{Goals}

\begin{Preparations}
\item Att göra.
\end{Preparations}

\subsection{Obligatoriska uppgifter}

\Task En labbuppgiftsbeskrivning.

\Subtask En underuppgift.

\Subtask En underuppgift.

\subsection{Frivilliga extrauppgifter}

\Task En labbuppgiftsbeskrivning.

\Subtask En underuppgift.

\Subtask En underuppgift.
    
%!TEX encoding = UTF-8 Unicode

%!TEX root = ../compendium.tex

\input{generated/w09-chaphead-generated.tex}
    
%!TEX encoding = UTF-8 Unicode

%!TEX root = ../compendium.tex

\Exercise{\ExeWeekNINE}

\begin{Goals}
\item 
\end{Goals}

\begin{Preparations}
\item 
\end{Preparations}

\BasicTasks %%%%%%%%%%%%%%%%

\Task 
\begin{REPL}
scala> class Cell[T](var x: T){
         val typeName: String = x.getClass.getTypeName
         override def toString = "[" + typeName + "](" + x + ")"
       }
\end{REPL}

\Subtask 

\ExtraTasks %%%%%%%%%%%%%%%%%%%

\Task 

\AdvancedTasks %%%%%%%%%%%%%%%%%

\Task     
    
%!TEX encoding = UTF-8 Unicode

%!TEX root = ../compendium.tex

\Lab{\LabWeekNINE}

\begin{Goals}
\item Att lära sig.
\end{Goals}

\begin{Preparations}
\item Att göra.
\end{Preparations}

\subsection{Obligatoriska uppgifter}

\Task En labbuppgiftsbeskrivning.

\Subtask En underuppgift.

\Subtask En underuppgift.

\subsection{Frivilliga extrauppgifter}

\Task En labbuppgiftsbeskrivning.

\Subtask En underuppgift.

\Subtask En underuppgift.
    
%!TEX encoding = UTF-8 Unicode

%!TEX root = ../compendium.tex

\input{generated/w10-chaphead-generated.tex}
    
%!TEX encoding = UTF-8 Unicode

%!TEX root = ../compendium.tex

\Exercise{\ExeWeekTEN}

\begin{Goals}
\item 
\end{Goals}

\begin{Preparations}
\item 
\end{Preparations}

\BasicTasks %%%%%%%%%%%%%%%%

\Task 

\Subtask 

\ExtraTasks %%%%%%%%%%%%%%%%%%%

\Task 

\AdvancedTasks %%%%%%%%%%%%%%%%%

\Task     
    
%!TEX encoding = UTF-8 Unicode

%!TEX root = ../compendium.tex

\Lab{\LabWeekTEN}

\begin{Goals}
\item Att lära sig.
\end{Goals}

\begin{Preparations}
\item Att göra.
\end{Preparations}

\subsection{Obligatoriska uppgifter}

\Task En labbuppgiftsbeskrivning.

\Subtask En underuppgift.

\Subtask En underuppgift.

\subsection{Frivilliga extrauppgifter}

\Task En labbuppgiftsbeskrivning.

\Subtask En underuppgift.

\Subtask En underuppgift.
    
%!TEX encoding = UTF-8 Unicode

%!TEX root = ../compendium.tex

\input{generated/w11-chaphead-generated.tex}
    
%!TEX encoding = UTF-8 Unicode

%!TEX root = ../compendium.tex

\Exercise{\ExeWeekELEVEN}

\begin{Goals}
\item 
\end{Goals}

\begin{Preparations}
\item 
\end{Preparations}

\BasicTasks %%%%%%%%%%%%%%%%

\Task 

\Subtask 

\ExtraTasks %%%%%%%%%%%%%%%%%%%

\Task 

\AdvancedTasks %%%%%%%%%%%%%%%%%

\Task     
    
%!TEX encoding = UTF-8 Unicode

%!TEX root = ../compendium.tex

\Lab{\LabWeekELEVEN}

\begin{Goals}
\item Att lära sig.
\end{Goals}

\begin{Preparations}
\item Att göra.
\end{Preparations}

\subsection{Obligatoriska uppgifter}

\Task En labbuppgiftsbeskrivning.

\Subtask En underuppgift.

\Subtask En underuppgift.

\subsection{Frivilliga extrauppgifter}

\Task En labbuppgiftsbeskrivning.

\Subtask En underuppgift.

\Subtask En underuppgift.
    
%!TEX encoding = UTF-8 Unicode

%!TEX root = ../compendium.tex

\input{generated/w12-chaphead-generated.tex}
    
%!TEX encoding = UTF-8 Unicode

%!TEX root = ../compendium.tex

\Exercise{\ExeWeekTWELVE}

\begin{Goals}
\item 
\end{Goals}

\begin{Preparations}
\item 
\end{Preparations}

\BasicTasks %%%%%%%%%%%%%%%%

\Task 

\Subtask 

\ExtraTasks %%%%%%%%%%%%%%%%%%%

\Task 

\AdvancedTasks %%%%%%%%%%%%%%%%%

\Task     
    
%!TEX encoding = UTF-8 Unicode

%!TEX root = ../compendium.tex

\Lab{\LabWeekTWELVE}

\begin{Goals}
\item Att lära sig om hur man separerar beteende från vy med hjälp av Model-View uppdelningen.
\item Att lära sig om grundläggande cellulära automata \eng{cellular automata}.
\item Att lära sig om trådar, dvs. hur man gör för att köra kod \emph{"samtidigt"} som annan kod.
\item Att lära sig om hur man kan använda matriser för att lösa problem.
\item Att lära sig...
\item Att lära sig...
\end{Goals}

\begin{Preparations}
\item Att göra.
\end{Preparations}

\subsection{Obligatoriska uppgifter}

% Gör denna tasken till en preparation?
\Task Skapa en model som kan visas i vyn.

\Subtask En underuppgift.

\Subtask En underuppgift.


\Task Implementera modellens beteende enligt reglerna för Life.

\Subtask En underuppgift.

\Subtask En underuppgift.


\subsection{Frivilliga extrauppgifter}

\Task Implementera andra regler för cellulära automata.

    Det finns massor med regler för cellulära automata med sina egna intressanta beteenden och tillstånd.
    Gör den eller de du tycker verkar mest intressant!

    Fler regler kan finnas här: \url{https://en.wikipedia.org/wiki/Category:Cellular_automaton_rules}

    Nedan följer några roliga exempel som valts ut och anses lämpliga.

    \Subtask Implementera cyklisk cellulär automata.

        Denna typ av automata kallar cyklisk just för att det finns $N$ möjliga tillstånd och när tillståndet N nås så är "nästa" tillstånd $0$.

        Regeln för att en cell byter tillstånd ges av att om en granne har tillståndet exakt ett över cellens tillstånd så får cellen sin grannes tillstånd.

        För att få intressant beteende brukar man initialisera hela brädet så att varje cell får ett slumpvalt tillstånd.

        \url{https://en.wikipedia.org/wiki/Cyclic_cellular_automaton}

    \Subtask Implementera Wireworld.

        Wireworld är ett lite annorlunda då man i Wireworld designar "kretsar" inte helt olika de som finns i moderna datorer.

        I Wireworld kan man skapa komponenter som fungerar som dioder samt transistorer, och med dessa bygga logiska grindar.

        \url{https://en.wikipedia.org/wiki/Wireworld}


\Task En labbuppgiftsbeskrivning.

    \Subtask En underuppgift.

    \Subtask En underuppgift.


\Task Implementera spara och ladda.

    \Subtask Spara brädets tillstånd till ett format som kan både exporteras/sparas och importeras/laddas.

    \Subtask Ladda in det exporterade tillståndet.



\Task Alternativ vy: Kör programmet i webbläsaren med Scala.js

% Tidsuppskattning fungerar nog inte, varierar stort
%Tidsuppskattning: ?h

\Task Alternativ vy: Kör programmet på Android



%!TEX encoding = UTF-8 Unicode

%!TEX root = ../compendium.tex

\input{generated/w13-chaphead-generated.tex}
    
%!TEX encoding = UTF-8 Unicode
%%% EMPTY
%!TEX encoding = UTF-8 Unicode
%%% EMPTY
%!TEX encoding = UTF-8 Unicode

%!TEX root = ../compendium.tex

\input{generated/w14-chaphead-generated.tex}
    
%!TEX encoding = UTF-8 Unicode
%%% EMPTY
%!TEX encoding = UTF-8 Unicode
%%% EMPTY


\part{Appendix}
\appendix
\input{postchapters/vbox.tex}
\input{postchapters/terminal.tex}
\input{postchapters/edit.tex}
\input{postchapters/compile.tex}
\input{postchapters/document.tex}
\input{postchapters/ide.tex}
\input{postchapters/build.tex}
\input{postchapters/version-mgmt.tex}
\input{postchapters/keywords.tex}

\chapter{Lösningsförslag till övningar}
\foreach \n in {1,...,9}{%
  \input{modules/w0\n-solutions.tex}
}
\foreach \n in {10,...,14}{%
  \input{modules/w\n-solutions.tex}
}

\chapter{Ordlista}

\end{document}
